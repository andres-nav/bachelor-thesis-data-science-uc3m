\documentclass[oneside, 12pt, a4paper]{book}

\errorcontextlines=5 % Adjust the number to show more or fewer lines

\makeatletter
\def\input@path{{./templates/}}
\usepackage{thesis_uc3m}
\usepackage{eurosym}
\graphicspath{{./figures} {./templates/figures}}
\makeatother

\title{Applications of Autonomous Drones for Non-Terrestrial Networks in Remote Areas}
\author{Andrés Navarro Pedregal}
\degree{Data Science and Engineering and Telecommunication Technologies Engineering}
\graduationyear{2024~--~2025}
\supervisor{José Alberto Hernández Gutiérrez}
\placeandyear{Leganés, 2025 \\ Madrid, Spain}

\hypersetup{
  pdfauthor=Andres Navarro Pedregal,
  pdftitle=Final Thesis | Andres Navarro Pedregal | Data Science and Engineering,
  pdfsubject=Final Thesis,
  pdfkeywords={Final Thesis, Andres Navarro Pedregal, Data Science and Engineering},
}

\addbibresource{./bibliography.bib} % load the bibliography file
\addbibresource{./uav_parts.bib} % load the uav_parts file

\loadglsentries{acronyms} % load the acronyms file

\DeclareSIUnit{\inch}{in} % Define the inch unit

\begin{document}

% \setcounter{tocdepth}{0} % Show only parts and chapters in the table of contents

\frontmatter
\maketitle

\blankpage%
\chapter*{Abstract}

This bachelor thesis presents the design, implementation, and testing of an autonomous drone system for non-terrestrial networks in remote areas. The project aims to provide researchers with an open-source, cost-effective solution for studying and deploying drone-based networks in challenging environments. The system consists of a modular Unmanned Aerial Vehicle (UAV), a control station, and a reconnaissance platform. The UAV is designed to be easily customizable, using off-the-shelf components for accessibility and maintainability. It features autonomous flight capabilities and can carry various sensors for different applications. The control station allows real-time monitoring and control of the UAV, while the reconnaissance platform processes data collected by the drone using advanced computer vision and machine learning techniques. The system incorporates a 360-degree camera and on-board processing capabilities, enabling real-time object detection and tracking. Integration with large language models enhances the system's analytical capabilities, providing detailed insights about detected objects. Testing demonstrated the system's ability to fly autonomously, detect and analyze objects in real-time, and communicate data effectively over long distances. With a flight time of 30-40 minutes and a modular design, the developed system offers a versatile platform for applications such as surveillance, search and rescue, and environmental monitoring in remote areas.

\textbf{Keywords:} autonomous drones, non-terrestrial networks, remote areas, unmanned aerial vehicles, reconnaissance platform, object detection, deep learning

\blankpage%

\chapter*{Acknowledgments}
\begingroup
\let\clearpage\relax % This temporarily disables \clearpage so the Agradecimientos chapter is on the same page

First and foremost, I would like to express my sincere gratitude to my supervisor, Dr. José Alberto Hernández Gutiérrez, for the opportunity to work with him on this project. His guidance, expertise, and support have been invaluable throughout the development of this thesis.

I would also like to extend my thanks to Tomás Martínez Cortés for his assistance and insights, which have greatly contributed to this work.

My heartfelt appreciation goes to the Telematics Department of Universidad Carlos III de Madrid, especially to David Larrabeiti for his unwavering support since the beginning of my academic journey. I am grateful to all members of the department for creating a nurturing and stimulating environment for research and learning.

Finally, I would like to thank my colleague Carlos de Quinto for his collaboration and support throughout this process. Your camaraderie and intellectual contributions have been instrumental in the completion of this thesis.

To all those mentioned and to the many others who have supported me along the way, thank you for your encouragement, patience, and belief in my abilities. This work would not have been possible without you.

\chapter*{Agradecimientos}

En primer lugar, quiero expresar mi más sincero agradecimiento a mi supervisor, el Dr. José Alberto Hernández Gutiérrez, por la oportunidad de trabajar con él en este proyecto. Su orientación, experiencia y apoyo han sido invaluables durante todo el desarrollo de esta tesis.

También me gustaría extender mi gratitud a Tomás Martínez Cortés por su asistencia y perspectivas, que han contribuido enormemente a este trabajo.

Mi más profundo aprecio va para el Departamento de Telemática de la Universidad Carlos III de Madrid, especialmente a David Larrabeiti por su apoyo inquebrantable desde el inicio de mi trayectoria académica. Estoy agradecido a todos los miembros del departamento por crear un ambiente estimulante y enriquecedor para la investigación y el aprendizaje.

Finalmente, me gustaría agradecer a mi colega Carlos de Quinto por su colaboración y apoyo durante todo este proceso. Tu camaradería y contribuciones intelectuales han sido fundamentales para la realización de esta tesis.

A todos los mencionados y a los muchos otros que me han apoyado en el camino, gracias por su aliento, paciencia y fe en mis capacidades. Este trabajo no habría sido posible sin ustedes.

\endgroup

\blankpage%

\renewcommand{\contentsname}{Table of Contents}
\tableofcontents

\blankpage%

\listoffigures

\blankpage%

\listoftables

\blankpage%

\printglossary[type=\acronymtype,style=long, title=List of Acronyms]

\blankpage%

\mainmatter%

\oldpart{Introduction} % This is the first part, so it is not preceded by a new page

\chapter{Motivation}\label{ch:motivation}

The rapid evolution of cellular users in recent years, refer to \cref{fig:cellular_subscriptions}, has significantly increased the demand for high-speed, reliable data connectivity. This growing demand has placed substantial pressure on existing network infrastructures, requiring new solutions to enhance both capacity and coverage.\ \glspl{hetnet} have been proposed as an effective strategy to address these challenges by creating multi-layered networks that enable efficient data offloading \autocite{xu2021survey}. This improves both the capacity and coverage across the network. However, the dense deployment of these networks also increases energy consumption, which is undesirable in today's environmentally conscious and cost-sensitive world.

\begin{figure}
  \includegraphics{mobil_cellular_subscriptions_graph.png}
  \caption{Global Trends in Mobile Cellular Subscriptions (1960–2017): The graph illustrates the exponential growth of mobile subscriptions per 100 people across various regions over recent decades. \autocite{pahoInformationSystems}}\label{fig:cellular_subscriptions}
\end{figure}

Reducing power consumption while maintaining \gls{qos} requirements has therefore become a key objective in the deployment and operation of mobile networks \autocite{lopezperez2022survey}. In this context, \glspl{ntn} have emerged as a promising approach to complement \glspl{tn} and expand coverage to regions that have historically been underserved due to the prohibitive costs or logistical challenges of deploying terrestrial base stations \autocite{ahmmed2022digital}.

\glspl{ntn} leverage airborne platforms, including \gls{uav}, \glspl{hap}, and satellites, to act as relay nodes or base stations, providing connectivity to end-users across vast geographical areas. Their key advantage lies in the ability to cover expansive regions, including remote and inaccessible areas, where terrestrial solutions are either too costly or impractical. In particular, \gls{leo} satellites, which orbit at altitudes between 200 and 2000 kilometers, have shown significant potential for providing high-capacity connectivity due to their lower latency and stronger signal strength compared to other satellite types \autocite{giordani2021non}. This proximity to Earth not only enhances performance but also reduces energy requirements, aligning with the broader goal of minimizing power consumption in modern networks.

The emergence of \glspl{ntn} has further enabled the development of advanced concepts such as the \gls{istn} \autocite{ao2018space}\autocite{huang2019airplane}\autocite{liu2022operation}, and the \gls{iost} \autocite{akyildiz2019internet}\autocite{priyadarshini2022novel}\autocite{kak2021designing}. These concepts envision a seamless integration of terrestrial and non-terrestrial components to deliver next-generation communication services, particularly for future \gls{6g} networks. Mega-constellations of satellites, exemplified by networks like Starlink \autocite{tao2022impact} and OneWeb \autocite{zhu2022laser}, are at the forefront of this transformation. By integrating these networks with terrestrial systems, it becomes possible to connect isolated regions, including rural and oceanic environments, which are otherwise challenging to serve. Furthermore, this integration holds the potential to create a unified communication infrastructure that offers connectivity not only on the ground but also in the air and space.

This research is motivated by the growing need to develop sustainable, cost-effective solutions for extending connectivity to remote areas. \glspl{ntn}, particularly when integrated with \glspl{tn}, present a viable path forward in addressing these challenges, thereby supporting the goals of global connectivity and reducing energy consumption in future wireless communication systems. The work is supported by ongoing efforts in \glspl{ntn} to leverage emerging technologies for the development of \gls{6g} networks, with the ultimate aim of delivering ubiquitous, high-quality communication services to all corners of the world.

\chapter{Statement of the problem}
\label{ch:statement_of_problem}

Current research in the domain of \gls{ntn} primarily focuses on the development of satellite constellations in \gls{geo} and \gls{leo} \autocite{non_terrestial_networks_trends}. While these satellites offer significant potential, their high costs and limited customizability render them impractical for many research groups and individual researchers. Moreover, the complexity of launching and maintaining satellites in orbit presents a significant barrier to entry for many interested parties. In contrast, drones present a more affordable and adaptable solution for rapid deployment of communication networks in remote areas as well as low-altitude surveillance and monitoring missions.

The core challenge addressed in this thesis is the lack of widely accessible solutions for the deployment of autonomous drones in remote areas. Presently, available solutions are either expensive, proprietary systems that lack customization, or they require extensive infrastructure, such as ground stations or high-speed internet connectivity. These requirements severely limit the application of drones in areas with minimal or no infrastructure.

This work seeks to address the gap by developing a comprehensive, open-source, and cost-effective solution for autonomous drone missions in remote environments. Given the broad nature of this problem, the scope of this thesis will be narrowed to a specific environment and use case, as outlined below:

\begin{itemize}
  \item \textbf{Environment:} The modeled environment will be a remote area with minimal infrastructure, such as a forest, desert, or mountain. Specifically, the case study will involve an esplanade—a flat area devoid of significant obstacles like buildings or trees—allowing the drone to operate without the risk of collision. Moreover, \gls{4g} or \gls{5g} connectivity will be assumed to be available, enabling the drone to communicate with a ground station.

  \item \textbf{Atmospheric Conditions:} The selected environment will feature a clear sky, with minimal electromagnetic interference from other sources such as drones or aircraft. Additionally, the conditions will approximate \gls{stp} of \SI{15}{\degreeCelsius} and \SI{1013}{\hecto\pascal}.

  \item \textbf{Operational Parameters:} The drone operations will be confined to \gls{vlos} and an altitude below \SI{120}{\metre}, ensuring compliance with current aviation regulations in Spain and most countries. The \gls{mtow} of the drone will not exceed \SI{25}{\kilogram}.

  \item \textbf{Hardware:} The drone will be constructed using commercially available, off-the-shelf components to ensure affordability and ease of replication for other research groups and individuals.

  \item \textbf{Application:} This work will focus on the application of drones in human rescue missions, such as search and rescue operations in disaster-stricken areas or monitoring remote environments for signs of danger. While there are other potential applications of drones in remote areas, human rescue offers significant societal benefits.
\end{itemize}

\chapter{Objectives}
\label{ch:objectives}

The primary aim of this thesis is to provide the research community and humanitarian organizations with an open-source, modular, and customizable drone capable of operating in remote areas. The drone will be designed to be cost-effective, user-friendly, and competitive with other market offerings. It will incorporate the latest advancements in drone technology to ensure optimal performance and reliability.

In addition, a software platform will be developed to enable the operation of the drone for reconnaissance tasks. The platform will support multiple drones, allowing for the deployment of a coordinated swarm to conduct surveillance and monitoring missions effectively. The software will be user-friendly, enabling researchers and humanitarian organizations to program the drones for specific tasks easily.

To achieve these objectives, the following specific aims will be pursued:

\begin{itemize}
  \item The design must be modular and customizable, enabling easy modifications to adapt the drone for different applications.

  \item The components utilized in the drone should be off-the-shelf and readily available, facilitating straightforward assembly and repairs.

  \item The drone must be capable of autonomous flight to enable operations in remote areas where manual control is challenging.

  \item The design will incorporate the latest advancements in drone technology, ensuring competitiveness with other market offerings.

  \item The drone must be capable of communicating with a ground station to facilitate remote control.

  \item The software platform should be programmable to perform specific tasks, such as reconnaissance of designated areas and monitoring of particular parameters.

  \item The software platform must support multiple drones, allowing for the deployment of a coordinated swarm to conduct reconnaissance tasks effectively.
\end{itemize}

\chapter{Socio-economic environment}\label{ch:socio_economic_environment}

The widespread adoption of \glspl{uav} is transforming society and the economy, offering significant benefits while also presenting challenges. This chapter examines the socio-economic factors influencing the deployment of autonomous drones, particularly in \glspl{ntn} for remote areas. It explores the potential advantages, obstacles, and financial implications of integrating this technology across industries.

\section{Social Impact}\label{sec:social_impact}

Autonomous \glspl{uav} have gained popularity due to their versatility and efficiency, revolutionizing fields like agriculture, construction, and public safety. Their ability to perform complex tasks quickly can greatly enhance societal well-being. For example, \glspl{uav} equipped with thermal cameras improve emergency response by locating individuals in burning buildings, while high-resolution cameras can monitor large events. The project demonstrated how \glspl{uav} can provide real-time reconnaissance, aiding first responders and improving public safety.

In remote regions, autonomous \glspl{uav} facilitate connectivity by delivering essential supplies such as medicine and food. This capability not only improves the quality of life but also bridges the digital divide by providing internet access to underserved communities, thereby enhancing educational and informational resources.

However, privacy and security remain concerns. The extensive data collection by \glspl{uav} raises issues of surveillance, while the potential for misuse, such as spying or attacks, underscores the need for clear regulations and ethical guidelines, see \cref{ch:regulatory_framework}.

\section{Economic Impact}\label{sec:economic_impact}

The economic benefits of adopting autonomous \glspl{uav} are substantial, with the potential to boost productivity and drive innovation. In agriculture, \glspl{uav} enable precision farming through crop monitoring, pest detection, and optimized irrigation, helping farmers improve yields and reduce costs. The construction industry also benefits, as \glspl{uav} offer efficient infrastructure inspections, reducing manual labor and associated risks.

Despite these advantages, the costs of adopting \glspl{uav} (e.g., purchase, training, maintenance, and insurance) remain significant barriers for small companies and groups. Moreover, the need for skilled operators and the risk of accidents or malfunctions can increase operational expenses which is specially challenging for research groups in \glspl{ntn}.

\section{Environmental Impact}\label{sec:environmental_impact}

Autonomous \glspl{uav} can positively impact the environment by reducing fossil fuel usage. Their electric motors are more energy-efficient and produce fewer emissions than traditional vehicles, contributing to lower air pollution and greenhouse gas levels.

Nonetheless, the environmental costs of manufacturing and disposing of \glspl{uav} must be considered. The production of lightweight materials like carbon fiber is energy-intensive, and lithium-ion batteries pose recycling challenges. Sustainable design practices are necessary to mitigate these effects.

\section{Budget Analysis}\label{sec:budget_analysis}

As previously mentioned, the cost of adopting autonomous drones is a significant hurdle. This section breaks down the expenses associated with purchasing and operating drones, categorized into manufacturing, operating, and additional costs. The analysis includes a detailed assessment of hardware components, maintenance, insurance, and other relevant expenses.

Tables \cref{tab:manufacturing_costs_uav}, \cref{tab:manufacturing_costs_reconnaissance_platform}, \cref{tab:operating_costs}, and \cref{tab:other_costs} provide comprehensive cost breakdowns. The total cost, encompassing all categories, offers a clear picture of the financial investment needed for adopting drones in \glspl{ntn} applications. For this project, the total cost amounts to 6316 \euro, which includes the necessary hardware, software, and operational expenses. However, it does not account for human-related costs, such as training and salaries, which are essential for long-term sustainability.

In conclusion, while autonomous drones hold significant promise for remote areas and various industries, careful consideration of social, economic, and environmental factors is essential. Addressing these challenges with appropriate regulations and sustainable practices will maximize their potential benefits.

\begin{table}[H]
  \begin{tabular}{ l l r r }
    \toprule
    \textbf{Item} & \textbf{Model} & \textbf{Quantity} & \textbf{Cost (\euro)} \\
    \midrule
    Airframe & Tarot XS690 \autocite{rcinnovationsQuadPlegable} & 1 & 199 \\
    Motors & Tmotor U7 V2 420KV \autocite{rcinnovationsTmotor420KV} & 4 & 520 \\
    \gls{esc} & Tmotor FLAME 100A LV 600Hz \autocite{rcinnovationsVariadorTmotor} & 4 & 360 \\
    Propellers & Tmotor 17$\times$5.8 V2 \autocite{rcinnovationsTmotor17x58} & 2 & 144 \\
    Flight Controller & Holybro Pixhawk 6C \autocite{rcinnovationsPixhawkCarcasa} & 1 & 207 \\
    Battery & TATTU 22000mAh 4S 14.8V 30C \autocite{rcinnovationsComprarBatera}  & 1 & 270 \\
    \gls{pdb} & Holybro PM07 Power Module \autocite{rcinnovationsHolybroPM07} & 1 & 48 \\
    Regulator & HobbyWing 25A HV UBEC \autocite{rcinnovationsHobbyWingUbec} & 1 & 58 \\
    \gls{gps} & Holybro M9N GPS GNSS \autocite{rcinnovationsHolybroGNSS} & 1 & 70 \\
    Radiocontroller & RadioMaster TX16S \autocite{rcinnovationsRadioMasterTX16S} & 1 & 200 \\
    \gls{rf} Module & RFD868 TXMOD V2 868Mhz 1W \autocite{rcinnovationsComprarMdulos} & 1 & 423 \\
    Miscellaneous & Screws, Nuts, Wires, etc. &~--~& 100 \\
    \midrule
    \textbf{Total} & & & 2599 \\
    \bottomrule
  \end{tabular}
  \caption{Manufacturing costs for the \gls{uav}.}\label{tab:manufacturing_costs_uav}
\end{table}

\begin{table}[H]
  \begin{tabular}{ l l l r }
    \toprule
    \textbf{Item} & \textbf{Model} & \textbf{Quantity} & \textbf{Cost (\euro)} \\
    \midrule
    On-board Computer & NVIDIA Jetson Orin \autocite{nvidiaNVIDIAJetson} & 1 & 831 \\
    Camera & Ricoh Theta X 360 Degree Camera \autocite{ricohimagingTHETARicoh} & 1 & 800 \\
    Router & GL.iNet GL-X750 \autocite{glinetGLX750Spitz} & 1 & 153  \\
    4G SIM Card & Orange Prepaid SIM Card (1 month) & 1 & 10 \\
    Server & Amazon Web Services (1 month) & 1 & 10 \\
    \midrule
    \textbf{Total} & & & 1804 \\
    \bottomrule
  \end{tabular}
  \caption{Manufacturing costs for the reconnaissance platform} \label{tab:manufacturing_costs_reconnaissance_platform}
\end{table}

\begin{table}[H]
    \begin{tabular}{ l l r }
        \toprule
        \textbf{Item} & \textbf{Description} & \textbf{Cost (\euro)} \\
        \midrule
        Insurance & Liability Insurance for Pilots & 50 \\
        Flying Field & Flying Club Membership (1 year) & 350 \\
        Licensing & Drone Pilot License (1 year) & 0 \\
        Software & ArduPilot Software License (1 year) & 0 \\
        Maintenance & Spare Parts & 50 \\
        \midrule
        \textbf{Total} & & 450 \\
        \bottomrule
    \end{tabular}
    \caption{Operating costs.}\label{tab:operating_costs}
\end{table}

\begin{table}[H]
    \begin{tabular}{ l l r r }
        \toprule
        \textbf{Item} & \textbf{Model} & \textbf{Quantity} & \textbf{Cost (\euro)} \\
        \midrule
        Battery Charger & ISDT K4 Dual Charger \autocite{rcinnovationsISDTCargador} & 1 & 200 \\
        Lipo Bags & Lipo Safe Bags (Large) \autocite{rcinnovationsBolsaProtectora} & 1 & 13 \\
        Tools & Screwdriver Set, Pliers, etc. &~--~& 50 \\
        Solders & Soldering Iron, Solder, etc. &~--~& 200 \\
        3D Printer & Bamboo P1S \autocite{bambulabBambuPrinter} & 1 & 1015 \\
        3D Filament & PLA Filament (1kg) & 1 & 30 \\
        \midrule
        \textbf{Total} & & & 1508 \\
        \bottomrule
    \end{tabular}
    \caption{Other costs.}\label{tab:other_costs}
\end{table}

\chapter{Regulatory Framework}\label{ch:regulatory_framework}

The regulatory framework governing drones is a complex and dynamic area, influenced by various laws and regulations that differ from country to country. Generally, drone operations are regulated by aviation authorities responsible for ensuring safe and responsible usage.

\section{Relevant Institutions}

\subsection{European Union Aviation Safety Agency (EASA)}
The \gls{easa} \autocite{eu-1139-2018} plays a crucial role in harmonizing aviation safety standards across all \gls{eu} member states. Its primary objective is to maintain a consistent and high level of safety in civil aviation operations throughout the \gls{eu}.\ \gls{easa} achieves this through the establishment and enforcement of common regulations applicable to all member states. Notably, for the standardization of \gls{uas}, \gls{easa} has implemented Regulations (EU) 2019/947 \autocite{eu-947-2019} and (EU) 2019/945 \autocite{eu-945-2019}.

\subsection{State Aviation Safety Agency (AESA)}
In Spain, the \gls{aesa} \autocite{sp-184-2008} serves as the national regulatory authority, overseeing compliance with civil aviation standards within the aerospace sector.\ \gls{aesa} plays a critical role in promoting the development and application of aviation legislation, ensuring that the Spanish civil aviation system upholds the highest safety, quality, and sustainability standards. In instances of non-compliance with aviation regulations, \gls{aesa} possesses the authority to enforce sanctions.

\section{Applicable Legislation}

\subsection{Implementing Regulation (EU) 2019/947}
The Implementing Regulation (EU) 2019/947 \autocite{eu-947-2019} establishes the operational rules and requirements for UAS within the \gls{eu}. It provides a legal framework for the utilization of \gls{uas} across various operational categories, outlining requirements for operational authorizations and risk assessments where applicable. The regulation sets standards for remote pilot competency, operational procedures, and safety management to conduct \gls{uas} flights safely and effectively.

Additionally, it integrates with the Delegated Regulation (EU) 2019/945 \autocite{eu-945-2019} by defining operational requirements related to the \gls{uas} classes established within it. The regulation details specific operational limitations and conditions for each \gls{uas} class, including the management of \gls{uas} in classes C0 through C4. It also includes provisions for the safe integration of newly introduced \gls{uas} classes under Delegated Regulation (EU) 2020/1058 \autocite{eu-1058-2020}, specifically classes C5 and C6.

Moreover, this regulation addresses the procedures for \gls{uas} operators from third countries (non-\gls{easa} member states) wishing to operate within the \gls{ses} airspace, ensuring alignment with \gls{eu} standards and safety regulations.

\subsection{Delegated Regulation (EU) 2019/945}
The Delegated Regulation (EU) 2019/945 \autocite{eu-945-2019} defines the rules and standards for \gls{uas} within the \gls{eu}. It specifies the types of \gls{uas} that require certification regarding design, production, and maintenance. This regulation also provides guidelines for the commercialization of \gls{uas} intended for use in the Open category, as well as for remote identification accessories (e.g., Drone Remote ID). Furthermore, it outlines the requirements for the design and manufacture of \gls{uas} intended for operations defined in the Implementing Regulation (EU) 2019/947.

\subsection{Regulation (EU) 2024/1689: Artificial Intelligence Act}
The \gls{ai_act} of the \gls{eu} \autocite{AIActIntoForce}, which came into force on the 1st of August 2024, aims to ensure that \gls{ai} systems are safe, transparent, and ethical, while fostering innovation and protecting fundamental rights as stated in the Delegated Regulation (EU) 2024/1689 \autocite{eu-1689-2024}. The \gls{ai_act} categorizes \gls{ai} systems by risk, imposing strict requirements on high-risk applications, particularly in aviation, which may affect public safety and fundamental rights. These requirements encompass robust risk management, transparency, human oversight, and data governance, ensuring that \gls{ai} systems are reliable and secure.

The \gls{ai_act} introduces significant compliance obligations that could escalate development costs and timelines. High-risk systems must adhere to stringent standards to access the \gls{eu} market, potentially challenging innovation but ultimately aiming to build trust and facilitate broader adoption of \gls{ai} technologies within the \gls{eu}.

\section{Operational Categories}
The Regulation (EU) 2019/947 \autocite{eu-947-2019} classifies \gls{uas} into three distinct categories:

\begin{itemize}
  \item \textbf{Open Category:} The least restrictive category, designed for low-risk operations, includes activities such as recreational flying and commercial operations posing minimal risk to people and property. Operators must adhere to specific limitations (e.g., flying below \SI{120}{\meter}, maintaining \gls{vlos}).\ \gls{uas} must weigh under \SI{25}{\kilogram}, and pilots must ensure that the drone does not fly over people or in restricted areas. No prior authorization is required, though registration and remote pilot training are compulsory for all operations, except for drones weighing less than \SI{250}{\gram} that lack a camera or sensor.

  \item \textbf{Specific Category:} This category covers medium-risk operations necessitating a more detailed assessment. It includes operations that may involve flying over people or in restricted areas, provided mitigation procedures are in place. Operators must conduct a risk assessment and obtain an operational authorization known as \gls{sts} from \gls{aesa}. Requirements for \gls{uas} and pilot qualifications may vary based on the specific risk assessment and operational procedures defined within it.

  \item \textbf{Certified Category:} Designed for high-risk operations, this category involves stringent requirements comparable to those for manned aviation.\ \gls{uas} must meet specific certification standards and operators must comply with strict safety regulations. This category often includes advanced training requirements and operational procedures similar to those for commercial air transport.
\end{itemize}

\subsection{Open Category}
This work will focus on civil \gls{uas} that fall under \gls{easa}'s Open Category, although some findings may be applicable to other categories with appropriate regulatory adjustments. Within the Open Category, three subcategories differentiate based on associated risk, aircraft weight, and operational limits:

\begin{enumerate}
  \item \textbf{A1}: \gls{uas} with a \gls{mtow} of less than \SI{250}{\gram} that can fly over people but not over assemblies of people.

  \item \textbf{A2}: \gls{uas} with an \gls{mtow} of less than \SI{4}{\kilogram} that can fly close to people but must maintain a horizontal distance of \SI{30}{\meter} (\SI{5}{\meter} in low-speed configuration).

  \item \textbf{A3}: \gls{uas} with an \gls{mtow} of less than \SI{25}{\kilogram} that must maintain a horizontal distance of \SI{150}{\meter} from residential, commercial, industrial, or recreational areas.
\end{enumerate}

Check \cref{fig:eu_regulations_open_category_chart} for a visual representation of the Open Category subcategories.

\begin{figure}
  \includegraphics{eu_regulations_open_category_chart.png}
  \caption{\glsentryshort{eu} Regulations Open Category chart describing the subcategories A1, A2, and A3 with their respective operational limitations \autocite{ageagleEuropeanUnion}}\label{fig:eu_regulations_open_category_chart}
\end{figure}

Moreover, additional rules applicable to all three subcategories include:

\begin{itemize}
  \item The maximum height must not exceed \SI{120}{\meter} above ground level, as the lower limit for general aviation is \SI{150}{\meter}. This leaves only a \SI{30}{\meter} separation between manned aviation and UAS.\@

  \item Operators must always maintain \gls{vlos} unless the aircraft is in ``follow me'' mode or the pilot is using \gls{fpv} goggles.

  \item Operators must register if the \gls{uas} weighs more than \SI{250}{\gram} or if the aircraft is equipped with a camera or sensor.

  \item The aircraft must possess a remote identification ID, which is standard in all C1-C6 categories, with the exception of C4 and privately built aircraft.
\end{itemize}

% Local Variables:
% jinx-local-words: "aesa ageagleEuropeanUnion easa sts"
% End:

\chapter{Document Structure}\label{ch:document_structure}

The document structure of this thesis is organized into six main parts, each focusing on a specific aspect of the research on autonomous drones for Non-Terrestrial Networks (NTNs) in remote areas.

The Theoretical Background part, \cref{part:theoretical_background}, provides the foundational knowledge necessary to understand the core concepts of the thesis. It delves into non-terrestrial networks, explaining different types such as geostationary satellites, low-earth orbit satellites, and high-altitude platforms. This part also covers unmanned aerial vehicles, their types, and applications, as well as an overview of deep learning techniques relevant to the project.

The State of the Art part, \cref{part:state_of_the_art}, explores the historical development of non-terrestrial networks and UAVs, current types and characteristics of NTN platforms, and modern trends in the field. This part provides context for the research and highlights the gaps in existing solutions that this thesis aims to address.

The Methodology part, \cref{part:methodology}, forms the core of the thesis, detailing the requirements, design, and implementation of the proposed system. It covers the unmanned aerial vehicle design, control station setup, reconnaissance platform development, and communication system integration. This part provides a comprehensive overview of how the system was built and configured.

The Results part, \cref{part:results}, presents the outcomes of the project, including the testing procedures for individual components and the system as a whole. It discusses the achievements of the implemented system, as well as the challenges encountered during development and how they were overcome.

Finally, the Conclusions part, \cref{part:conclusions}, summarizes the key findings of the research, reflects on the objectives achieved, and proposes directions for future work. It also includes a discussion on the socio-economic impact of the project and an analysis of the relevant regulatory framework governing UAV operations.

\chapter{Methodology Approach}

\todo{need to rephrasea all this}
The methodology approach used in this thesis is based on the V-model as outlined by the International Council of System Engineering (INCOSE) standard for project development. The V-model is a rigorous and structured approach to project development that ensures that all aspects of the project are considered and that the project is completed on time and within budget. It is achieived by a thorough development process, facilitating clear validation and verification of initial requirements at each stage.

The methodology is divided into seven parts:

\begin{enumerate}
  \item Identification of Solution Requirements: Before taking any actions, it is crucial to understand what can be achieved and the rationale behind it. To this end, the author has focused on deriving valuable insights from the reviewed literature in the State of the Art (SOT A) and identifying critical scenarios. Subsequently, and in conjunction with the thesis objectives and the applicable regulatory framework, a list of solution requirements is created. This list will later serve as a benchmark for validating the proposed solution.

  \item Identification of System Requirements: Technical requirements are formulated to meet the previously established solution requirements. This includes a high-level overview of the proposed solution’s components, the rationale for their selection, and their interconnections.

  \item Identification of Component Requirements: Building on the high-level architecture of the solution, a more detailed approach is outlined for each component, considering their specific power and data transmission needs. This results in a detailed architecture of the solution.

  \item Manufacturing and Implementation: The proposed solution is manufactured using available tools, while simultaneously integrating the necessary electrical components.

  \item Component Verification: Each component’s functionality is verified in standalone mode, with detailed information provided on the verification process.

  \item Integration Testing and Flight Testing: The methodology for conducting flight tests and post-flight analysis is described. System integration is performed by checking communication between module pairs to ensure that data can travel freely and be used effectively.

  \item Deployment and Preparation for Future Upgrades: Validation of the initial requirements is conducted to ensure that all solution requirements have been addressed. This step also includes preparations for potential future upgrades.
\end{enumerate}


\oldpart{Theoretical Background}\label{part:theoretical_background}

\chapter{Non-Terrestrial Networks}
\label{ch:non_terrestrial_networks}

\glspl{ntn} represent a groundbreaking approach in wireless communication, utilizing aerial and space-based platforms to deliver connectivity services. These platforms can operate at varying altitudes, from a few hundred meters to several kilometers above ground, offering key advantages over traditional terrestrial networks, such as expanded coverage, increased capacity, and greater flexibility.

\glspl{ntn} provide a promising solution to meet the rising demand for wireless access in remote and underserved areas. By leveraging aerial and space platforms, \glspl{ntn} can extend the reach of conventional terrestrial networks, offering connectivity where traditional infrastructure is either difficult or impossible to deploy. These platforms are available in several configurations, including \glspl{hap}, \gls{leo} satellites, and \gls{geo} satellites, each with distinct benefits in terms of coverage, capacity, and latency, making them suitable for various use cases. In \cref{fig:ntn_types}, we illustrate the different types of \glspl{ntn} based on altitude and platform type as well as their interaction with other network elements such as ground stations, \glspl{uav}, \gls{iot} devices, etc.

\section{Geostationary Satellites}

\gls{geo} satellites operate at an altitude of approximately \SI{35786}{\kilo\meter} above the Earth's equator. These satellites maintain a fixed position relative to the Earth’s surface, as they orbit at the same rate as the planet's rotation. \gls{geo} satellites offer extensive coverage, often spanning entire continents, and are widely used for telecommunications, broadcasting, and weather monitoring.

Compared to \gls{leo} satellites, \gls{geo} satellites offer broader coverage and higher capacity, making them ideal for applications like direct-to-home television, satellite radio, and broadband internet access. Their capacity to cover large regions makes \glspl{geo} integral to global communications infrastructure.

A key advantage of \gls{geo} satellites is their ability to support high-capacity services for many users, making them valuable for broadcasting live events, such as sports and concerts, or delivering high-definition video content worldwide. As a fundamental part of the global media and communication ecosystem, \gls{geo} satellites are expected to continue playing a vital role in providing connectivity to remote and underserved regions.

\section{Low-Earth Orbit Satellites}

\gls{leo} satellites operate at altitudes between \SI{160}{\kilo\meter} and \SI{2000}{\kilo\meter} above the Earth. Orbiting at high speeds, these satellites provide global coverage, making them ideal for delivering connectivity to remote and underserved regions. \gls{leo} satellites offer several advantages over traditional \gls{geo} satellites, including lower latency, higher capacity, and reduced infrastructure costs.

Deployed in constellations consisting of hundreds or thousands of satellites, \gls{leo} satellites work together to provide continuous coverage. These satellites communicate through inter-satellite links, allowing data to be relayed seamlessly across the constellation. \gls{leo} satellites are well-suited for delivering broadband internet access in areas where traditional infrastructure is difficult to establish.

One of the primary advantages of \gls{leo} satellites is their low latency, enabling real-time communication and supporting applications that require minimal delay, such as online gaming, video conferencing, and autonomous vehicle systems. Additionally, \gls{leo} satellites offer high-speed internet connectivity to users in remote locations, granting access to online services, educational platforms, and e-commerce opportunities. As a critical part of the emerging \gls{ntn} ecosystem, \gls{leo} satellites are expected to play a key role in bridging the global digital divide.

\section{High-Altitude Platforms}

\glspl{hap} operate at altitudes ranging from a few hundred meters to several kilometers above the Earth. These platforms may consist of balloons, airships, or \glspl{uav}. They provide several benefits compared to traditional terrestrial networks, including broader coverage, increased capacity, and lower infrastructure costs. \glspl{hap} can be deployed swiftly to offer connectivity in remote or underserved areas, making them an effective tool for reducing the digital divide.

Additionally, \glspl{hap} can provide temporary connectivity in disaster-stricken regions or during large-scale events. Rapid deployment allows emergency responders to coordinate efforts efficiently. They can also extend the coverage of existing networks in rural areas, where conventional infrastructure is costly or challenging to install.

Advances in \gls{uav} technology have enabled the development of autonomous \glspl{hap} equipped with \gls{lte} or \gls{5g} base stations, flying at altitudes of up to \SI{20}{\kilo\meter}. These platforms cover vast areas and support a range of applications, making them ideal for remote and underserved locations where deploying traditional infrastructure is not practical.

\begin{figure}
  \includegraphics{non_terrestrial_networks_types.png}
  \caption{Types of \glsentryshortpl{ntn} based on altitude and platform type. \glsentryshortpl{hap}, \glsentryshortpl{leo} satellites, and \glsentryshortpl{geo} satellites interact with other network elements to provide connectivity services. \autocite{alertifyAirbusNTT}}
  \label{fig:ntn_types}
\end{figure}

\chapter{Unmanned Aerial Vehicles}\label{ch:unmanned_aerial_vehicles}

The term \gls{uav} encompasses a diverse range of aircraft, from compact quad-copters to larger fixed-wing models. What primarily distinguishes \glspl{uav} from conventional aircraft is that \glspl{uav} are either remotely piloted or autonomously controlled, whereas traditional aircraft are operated by human pilots. This autonomy enables \glspl{uav} to be deployed in scenarios where human presence is either impractical or hazardous, such as military operations or environments with significant risk.

\glspl{uav} serve various purposes across multiple industries, including surveillance, reconnaissance, search and rescue missions, and scientific research. Additionally, they have become invaluable in sectors like agriculture, forestry, and environmental monitoring. In recent years, \glspl{uav} have gained popularity among hobbyists for recreational flying and personal projects.

\section{Types of Unnamed Aerial Vehicles}\label{sec:uav_types}

In line with the classification presented by the \gls{easa} \autocite{eu-947-2019}, \glspl{uav} can be categorized into three main types, each based on size, weight, physical design, and operational capabilities:

\begin{itemize}
  \item \textbf{Fixed-wing \glspl{uav}:} These aircraft feature fixed wings and function similarly to conventional airplanes. They tend to be larger and designed for long-duration flights, making them suitable for extended missions like surveillance, reconnaissance, and mapping. However, they require runways for takeoff and landing, and cannot hover in place, limiting their versatility in confined spaces.

  \item \textbf{Rotary-wing \glspl{uav}:} Equipped with multiple rotors, these \glspl{uav} can take off and land vertically. They are typically smaller and more agile, making them ideal for close-range missions like aerial photography, search and rescue, and surveillance. However, their limited range and endurance compared to fixed-wing \glspl{uav} restrict their use in long-duration missions.

  \item \textbf{Hybrid \glspl{uav}:} Combining the features of both fixed-wing and rotary-wing designs, hybrid \glspl{uav} can take off and land vertically like rotary-wing models while achieving greater range and endurance through fixed-wing flight. These aircraft are ideal for missions requiring versatility, such as reconnaissance and mapping. Despite their advantages, they are more complex and costly to operate than single-type \glspl{uav}.
\end{itemize}

\Cref{fig:uav_types} illustrates the different \glspl{uav} types based on their design and intended applications.

\begin{figure}
  \includegraphics{uav_types.png}
  \caption{Types of \glsentryshortpl{uav} based on their design and intended use \autocite{mohnani2022role}.\ a. Rotary-wing \glsentryshort{uav}, b. Fixed-wing \glsentryshort{uav}, c. Hybrid \glsentryshort{uav}}\label{fig:uav_types}
\end{figure}

In \cref{tab:uav_categories}, a detailed comparison of fixed-wing, rotary-wing, and hybrid \glspl{uav} is presented across various performance metrics, including size, range, endurance, cost, and ease of operation. The table provides insights to help in selecting the appropriate \gls{uav} for specific mission requirements, depending on factors like payload, range, and maneuverability.

\begin{table}
  \begin{tabular}{ l c c c }
    \toprule
    \textbf{Metric} & \textbf{Fixed-wing} & \textbf{Rotary-wing} & \textbf{Hybrid} \\
    \midrule
    Size & Moderate & Small & Large \\
    Range & Long & Short & Moderate \\
    Endurance & High & Low & Moderate \\
    Payload capacity & High & Low & Moderate \\
    Maneuverability & Low & High & Moderate \\
    Ease of use & Moderate & High & Low \\
    Maintenance & Moderate & Low & High \\
    Runway requirement & Yes & No & Yes \\
    Cost & Moderate & Low & High \\
    \bottomrule
  \end{tabular}
  \caption{Comparison of fixed-wing, rotary-wing, and hybrid \glsentryshortpl{uav} across various performance metrics}\label{tab:uav_categories}
\end{table}

\section{Applications of Unmanned Aerial Vehicles}

\glspl{uav} have a wide range of applications across different industries, leveraging their versatility, maneuverability, and autonomy. Some common applications of \glspl{uav} include:

\begin{itemize}
  \item \textbf{Aerial photography and videography:} \glspl{uav} equipped with high-resolution cameras are used for capturing aerial images and videos for various purposes, including filmmaking, real estate, and landscape photography.

  \item \textbf{Agriculture:} \glspl{uav} are employed in precision agriculture to monitor crop health, assess soil conditions, and optimize irrigation and fertilization practices. They can provide valuable insights to farmers for improving crop yield and reducing resource wastage.

  \item \textbf{Search and rescue:} \glspl{uav} equipped with thermal imaging cameras and other sensors are used in search and rescue operations to locate missing persons, assess disaster-affected areas, and deliver essential supplies to remote locations.

  \item \textbf{Infrastructure inspection:} \glspl{uav} are utilized for inspecting critical infrastructure like bridges, power lines, and pipelines. They can access hard-to-reach areas and capture detailed images for assessing structural integrity and identifying maintenance needs.

  \item \textbf{Environmental monitoring:} \glspl{uav} are deployed for monitoring environmental parameters like air quality, water quality, and wildlife populations. They can collect data in remote or hazardous environments, providing valuable insights for conservation efforts and scientific research.

  \item \textbf{Disaster response:} \glspl{uav} play a crucial role in disaster response by providing real-time situational awareness, mapping affected areas, and coordinating emergency operations. They can assist in assessing damage, locating survivors, and delivering aid to disaster-stricken regions.

  \item \textbf{Military and defense:} \glspl{uav} are extensively used in military and defense applications for reconnaissance, surveillance, target acquisition, and combat operations. They offer a cost-effective and low-risk alternative to manned aircraft in high-risk environments.

    \item \textbf{Delivery services:} \glspl{uav} are increasingly being used for last-mile delivery of goods and services. Companies like Amazon and UPS are exploring the use of \glspl{uav} for delivering packages to customers in urban and rural areas.
\end{itemize}

% Local Variables:
% jinx-local-words: "easa uav"
% End:

\chapter{Deep Learning}\label{ch:deep_learning}

\gls{dl} is a subfield of \gls{ml} that focuses on the development of algorithms and models inspired by the structure and function of the human brain. These algorithms are designed to learn from data, identify patterns and relationships, and make predictions or decisions without explicit instructions.\ \gls{dl} algorithms are characterized by their ability to automatically discover and extract features from raw data, enabling them to perform complex tasks such as image recognition, speech recognition, and natural language processing.

\gls{dl} has revolutionized various industries and domains, including healthcare, finance, transportation, and entertainment. By leveraging the power of \gls{dl}, organizations can analyze large datasets, extract valuable insights, and automate complex tasks, leading to improved decision-making, enhanced user experiences, and optimized processes. From self-driving cars and virtual assistants to medical diagnostics and fraud detection, \gls{dl} is transforming the way we interact with technology and the world around us.

Furthermore, \gls{dl} plays a crucial role in enabling \gls{uav} autonomy, allowing drones to perform tasks such as navigation, obstacle avoidance, and object recognition without human intervention. By integrating \gls{dl} algorithms into \gls{uav} systems, researchers and developers can enhance the capabilities and efficiency of drones, enabling them to operate in complex environments and execute sophisticated missions.

\section{Deep Learning Techniques}

\gls{dl} encompasses a wide range of techniques and architectures that enable machines to learn from data and make decisions. Some of the most common \gls{dl} techniques include:

\begin{itemize}
  \item \textbf{\glspl{ann}:} \glspl{ann} are computational models inspired by the structure and function of the human brain. They consist of interconnected nodes, or neurons, organized in layers, with each neuron performing a simple computation.\ \gls{ann} can learn complex patterns and relationships in data through a process called back-propagation, where errors are propagated back through the network to adjust the model's parameters.\ \gls{ann} are used in a variety of tasks, such as classification, regression, and clustering.

  \item \textbf{\glspl{cnn}:} \glspl{cnn} are a type of \glspl{ann} designed for processing and analyzing visual data, such as images and videos. They use convolutional layers to extract features from input data, pooling layers to reduce spatial dimensions, and fully connected layers to make predictions.\ \gls{cnn} are widely used in image recognition, object detection, and image segmentation tasks.

  \item \textbf{\glspl{rnn}:} \glspl{rnn} are a type of \glspl{ann} designed for processing sequential data, such as time series, text, and speech. They have feedback connections that allow information to persist over time, enabling them to capture temporal dependencies in data.\ \gls{rnn} are used in natural language processing, speech recognition, and machine translation tasks.

  \item \textbf{\glspl{gan}:} \glspl{gan} are a type of a \gls{dl} model that consists of two neural networks, a generator and a discriminator, trained adversarially. The generator generates synthetic data samples, while the discriminator distinguishes between real and fake samples.\ \gls{gan} are used in image generation, style transfer, and data augmentation tasks.
\end{itemize}

These techniques form the foundation of \gls{dl} and are used in a wide range of applications across various domains, enabling machines to perform complex tasks and make intelligent decisions. In the context of \gls{ntn} and \glspl{uav}, \gls{dl} techniques can enhance network performance, optimize resource allocation, and enable autonomous operation, leading to more efficient and reliable systems. Furthermore, \gls{dl} can enable \glspl{uav} to perform tasks such as navigation, object detection, and mission planning with high accuracy and efficiency, making them valuable tools for a wide range of applications (e.g., surveillance, monitoring, and disaster response).


\oldpart{State of the art}\label{part:state_of_the_art}

\chapter{Historical Development}
\label{ch:historical_development}

\todo{write this chapter}

\chapter{Types \& Characteristics}\label{ch:types_technologies_characteristics}

% types of ntns
% https://arxiv.org/pdf/2103.09156

% nice graph to show the structure of ntn
% https://arxiv.org/pdf/1912.10226

% \todo{write this chapter}
% https://ieeexplore.ieee.org/stamp/stamp.jsp?tp=&arnumber=9861699, add different types of ntn and the challenges

% https://ieeexplore.ieee.org/stamp/stamp.jsp?tp=&arnumber=8869712
% types of areal platformas, todo add image from it


\glspl{ntn} are an emerging approach in the field of wireless communication that aims to expand network coverage beyond the reach of traditional terrestrial systems. By utilizing platforms such as satellites, \glspl{uav}, and \glspl{hap}, \glspl{ntn} can address the growing demand for global connectivity, particularly in remote or inaccessible areas.\ \glspl{ntn} offer a flexible and adaptable infrastructure, supporting a wide range of applications from disaster recovery to rural broadband access.

The following sections explore the various types of \glspl{ntn}, also referred as architectural models, highlighting their roles in enhancing communication networks, the benefits they offer, and the challenges they face. These architectures include platforms functioning as network users, relays, and base stations, along with mixed models that combine different approaches to maximize efficiency and coverage. The diagram of the different architectures of \gls{ntn} platforms according to their use case is shown in \cref{fig:ntn_platforms}.

\begin{figure}
  \begin{subfigure}{0.4\textheight}
    \includegraphics{ntn_use_case_platform_as_a_user.png}
    \caption{Diagram of NTN Platforms as Network Users where the NTN acts as another UE in the network.}\label{fig:ntn_platform_as_a_user}
  \end{subfigure}

  \begin{subfigure}{0.4\textheight}
    \includegraphics{ntn_use_case_platform_as_a_relay.png}
    \caption{Diagram of NTN Platforms as Relays where the NTN acts as a relay between the UE and the base station.}\label{fig:ntn_platform_as_a_relay}
  \end{subfigure}

  \begin{subfigure}{0.4\textheight}
    \includegraphics{ntn_use_case_platform_as_a_base_station.png}
    \caption{Diagram of NTN Platforms as Base Stations where the NTN acts as a base station for the UE.}\label{fig:ntn_platform_as_a_base_station}
  \end{subfigure}

  \begin{subfigure}{0.4\textheight}
    \includegraphics{ntn_use_case_platform_mixed.png}
    \caption{Diagram of Mixed Architecture Models where the NTN acts as a combination of the above.}\label{fig:mixed_architecture_models}
  \end{subfigure}

  \caption{Different architectures of NTN platforms according to their use case \autocite{evolution_ntn_from_5g_6g_survey}.}\label{fig:ntn_platforms}
\end{figure}

\section{NTN Platforms as Network Users}\label{sec:ntn_platform_as_a_user}

In this architecture, the \gls{ntn} platform operates similarly to a user device, or \gls{ue}, connecting to an existing terrestrial network as depicted in \cref{fig:ntn_platform_as_a_user}. This model is particularly relevant for platforms like satellites and \glspl{uav}, which need connectivity for data transmission. \glspl{uav}, for instance, can access terrestrial networks via base stations located on the ground.

A well-known example is the integration of \glspl{uav} into terrestrial networks. The \gls{3gpp} has identified \glspl{uav} as a unique category of \gls{ue} \autocite{muruganathan2019overview3gpprelease15study}, leading to research on how to address their specific connectivity challenges \autocite{uav_assisted_networks_challenges}. These challenges include maintaining stable connections during flight, managing potential interference, and ensuring service quality at various altitudes. By treating \glspl{uav} as network users, terrestrial infrastructure can support a broader range of communication needs, such as surveillance, logistics, and remote sensing.

Additionally, in scenarios where terrestrial ground stations are unavailable or impractical, satellites can communicate directly with other satellites. This capability eliminates the need for extensive ground infrastructure, improving data transmission efficiency and making it particularly beneficial for space missions and remote observation activities.

\section{NTN Platforms as Relays}\label{sec:ntn_platform_as_a_relay}

\gls{ntn} platforms can also serve as relays, functioning as intermediaries that transmit signals between different components of the network, refer to \cref{fig:ntn_platform_as_a_relay}. This type of architecture can be categorized into two main configurations based on the relay's role within the network.

The first configuration involves backhaul connectivity \autocite{Elamassie2023FreeSO}, where the \gls{ntn} platform provides a link between a ground-based base station and the core network. This model is especially useful in remote or hard-to-reach areas where traditional backhaul solutions, such as fiber optic cables, are unavailable or too costly to install. By using satellites or \glspl{hap} as relays, terrestrial networks can be extended without requiring significant ground infrastructure.

In the second configuration, the \gls{ntn} platform acts as a direct relay between ground users and a terrestrial base station. This setup is particularly effective in environments where access to a base station is limited, such as densely populated urban areas or mountainous regions.\ \glspl{uav} or \gls{leo} satellites can serve as intermediary nodes, relaying user signals to the terrestrial network. This approach expands coverage and improves connectivity in areas where traditional infrastructure may not suffice \autocite{overview_on-5g-and-beyon-networks-with-uav}.

\section{NTN Platforms as Base Stations}\label{sec:ntn_platform_as_a_base_station}

\gls{ntn} platforms can also be configured to act as base stations, managing communication networks autonomously. Platforms equipped with advanced processing capabilities, such as regenerative payloads, can process signals onboard and provide connectivity without relying on ground-based infrastructure, as shown in \cref{fig:ntn_platform_as_a_base_station}. This architecture is particularly useful in scenarios where ground-based base stations are impractical or unavailable.

In the case of satellite-based base stations, satellites in \gls{leo} are equipped with onboard processors that handle tasks typically managed by terrestrial base stations, such as signal processing, traffic routing, and user management \autocite{leo_platforms_in_ntn}. This configuration is particularly well-suited for providing connectivity in remote regions, such as open seas, deserts, or disaster-stricken areas where ground infrastructure is impractical or unavailable.

\glspl{uav} can also function as temporary base stations, providing localized network coverage in specific areas. For example, \glspl{uav} equipped with \gls{5g} technology can deliver communication services during large-scale events, emergencies, or military operations. These \gls{uav}-based stations operate at lower altitudes than satellites, making them ideal for providing real-time, localized coverage with minimal delay.

\section{Mixed Architecture Models}\label{sec:mixed_architecture_models}

In real-world deployments, combining different architectural models is often necessary to create a more flexible and robust network infrastructure \autocite{hybrid_satellite_terrestrial_networks}, as shown in \cref{fig:mixed_architecture_models}. Mixed architecture models leverage the unique strengths of various \gls{ntn} platforms to optimize communication performance and coverage.

One example of a mixed architecture is a combination of \gls{leo} satellites and \glspl{uav}. In this configuration, a \gls{leo} satellite equipped with base station capabilities can work alongside \glspl{uav} acting as relays to extend coverage to ground users. The \glspl{uav}, operating at lower altitudes, enhance connectivity by relaying signals to the satellite, which then forwards the data to the core network. This approach ensures a broader coverage area and improved network performance.

Another example involves multi-tier satellite configurations \autocite{5g_allstar}, where satellites at different altitudes work together to provide comprehensive coverage. For instance, a \gls{geo} satellite can serve as a high-level backhaul link to the core network, while \gls{leo} satellites deliver low-latency connectivity to end users. This multi-tier approach combines the strengths of \gls{geo} and \gls{leo} satellites, offering both extensive coverage and low-latency communication.

Finally, \glspl{hap} can be integrated into \gls{ntn} networks to act as intermediary nodes between UAVs and satellites. In this configuration, \glspl{hap} receive data from \glspl{uav} below and transmit it to \gls{leo} satellites above, improving data transmission efficiency. This multi-hop communication strategy is particularly useful in complex environments where direct satellite or terrestrial communication is difficult.

\section{Characteristics of Non-Terrestrial Networks}\label{sec:characteristics_of_non_terrestrial_networks}

\glspl{ntn} offer several benefits, but they also present challenges. One of the primary advantages of \glspl{ntn} is their ability to extend network coverage to regions where terrestrial infrastructure is either unavailable or impractical \autocite{evolution_ntn_from_5g_6g_survey}.\ \glspl{ntn} can provide connectivity in remote areas, including oceans, mountainous regions, and locations affected by natural disasters. Platforms like \glspl{uav} and \glspl{hap} are especially useful in emergencies, as they can be quickly deployed to restore communication services and support disaster relief efforts.

In addition, \glspl{ntn} provide a flexible communication infrastructure by integrating different types of platforms. This adaptability makes \glspl{ntn} suitable for meeting the needs of diverse applications across various regions. Moreover,\ \glspl{ntn} are often more cost-effective than traditional networks, especially in sparsely populated areas where building extensive ground infrastructure would be prohibitively expensive.

However, \glspl{ntn} also face several challenges. Latency is a significant concern, especially in systems that rely on multiple satellite hops or satellite-to-satellite communication. High latency can affect the performance of real-time applications, such as voice or video communication. Signal interference is another issue, as multiple platforms operating at different altitudes and frequencies can lead to overlapping signals. Effective spectrum management is critical to maintaining service quality in \gls{ntn} deployments.

Power limitations, particularly for \glspl{uav} and certain \glspl{hap}, can restrict their operational duration \autocite{power_efficiency_uav}. Additionally, regulatory challenges, such as airspace management and frequency allocation, pose obstacles to the widespread deployment of \glspl{ntn}. For instance, \gls{uav}-based \gls{ntn} platforms must comply with international airspace regulations, while satellite-based \glspl{ntn} require coordination across different countries to ensure proper frequency usage.

\glspl{ntn} serve a variety of use cases across different sectors. In disaster recovery scenarios, \glspl{ntn} can be rapidly deployed to restore communication networks for emergency response teams \autocite{ntn_challenges_and_opportunities}.\ \glspl{ntn} are also valuable for environmental monitoring and remote sensing, providing continuous observation over large areas, such as forests, oceans, and agricultural lands. In the aviation and maritime sectors, \glspl{ntn} provide reliable connectivity for aircraft and ships, offering essential communication services in regions beyond terrestrial coverage. Finally, \glspl{ntn} play a crucial role in reducing the digital divide by delivering broadband internet access to rural and remote communities where traditional networks are not viable.

% Local Variables:
% jinx-local-words: "ntn uav ue"
% End:

\chapter{Modern Trends}\label{ch:modern_trends}

In recent years, \glspl{ntn} have emerged as a significant advancement in the field of wireless communication. These networks utilize platforms as described in previous chapters (e.g., satellites, \glspl{hap}, or \glspl{uav}), to provide connectivity in areas where traditional communication infrastructure may be lacking or underdeveloped. As the demand for reliable communication continues to grow, \glspl{ntn} present unique opportunities for various applications, particularly in disaster response, virtual reality, and high-speed internet access. This section explores the modern trends of \glspl{ntn}, highlighting their importance in enhancing \gls{uav} capabilities and addressing contemporary challenges.

One of the most impactful applications of \gls{ntn}-enabled \glspl{uav} is in rescue operations during natural disasters. These drones can serve as critical tools in emergencies such as earthquakes, hurricanes, and floods by establishing essential communication links between affected areas and external agencies. For instance, during the 2015 Nepal earthquake \autocite{drone_nepal_2015}, \glspl{uav} equipped with \gls{ntn} technology were deployed to transmit real-time video feeds of devastated regions, allowing emergency responders to assess damage and prioritize resources effectively. Additionally, \glspl{uav} can be outfitted with thermal imaging cameras to locate survivors in disaster zones. In the aftermath of Hurricane Harvey in 2017 \autocite{Greenwood2020FlyingIT}, \gls{ntn}-enabled \glspl{uav} scanned flooded neighborhoods, relaying findings to command centers and guiding rescue teams to those in need. The integration of \glspl{ntn} with \glspl{uav} thus creates a network of aerial assets that optimizes response times and enhances overall disaster management efforts.

Beyond rescue operations, \gls{ntn}-enabled \glspl{uav} are being utilized in innovative applications such as \gls{vr} experiences and \gls{5g} connectivity. By transmitting high-definition video feeds, these drones offer immersive experiences that allow users to engage with their surroundings in unique ways. For instance, during training scenarios for emergency responders, \glspl{uav} can capture 360-degree videos of disaster zones, providing trainees with a realistic simulation of on-ground conditions \autocite{vr_model_drone_footage_disaster_zone}. This not only enhances training effectiveness but also helps responders make informed decisions based on a comprehensive understanding of the situation. Moreover, companies like Nature Eye \autocite{natureeyeNatureEyeExplore} employ \gls{ntn}-enabled \glspl{uav} to provide live aerial tours of scenic locations, broadening the scope of \gls{vr} applications and enhancing user engagement.

As \gls{ntn} technology continues to evolve, the integration of \gls{ai} within \glspl{uav} presents opportunities for enhanced operational efficiency and autonomy. \gls{ai} algorithms can optimize flight paths, improve communication protocols, and enhance decision-making processes for \glspl{uav} in real-time. Companies like Skydio \autocite{skydioSkydioAutonomous} leverage \gls{ai} to enable their \glspl{uav} to autonomously navigate complex environments and avoid obstacles. Moreover, \gls{ml} techniques can enhance \glspl{uav}' ability to analyze data collected from their surroundings. Notably, researchers from the University of Zurich developed an autonomous drone system called Swift \autocite{Kaufmann2023ChampionlevelDR}, which can outperform human champions in first-person view (FPV) drone racing. This achievement represents a significant milestone, highlighting the potential of \gls{ai} in real-time decision-making and its applications in sectors such as environmental monitoring and precision agriculture.

In conclusion, the modern trends of \glspl{ntn} in \gls{uav} applications demonstrate the transformative impact these technologies can have across various sectors. From enhancing emergency response capabilities to revolutionizing communication infrastructure, \gls{ntn}-enabled \glspl{uav} are poised to play a critical role in shaping the future of wireless communication. Continued research and innovation in this field will unlock new possibilities and applications, ultimately contributing to the advancement of non-terrestrial networks and their integration with terrestrial systems.

% Local Variables:
% jinx-local-words: "ntn uav"
% End:


\oldpart{Methodology}\label{part:methodology}

\chapter{Requirements}\label{ch:requirements}

Based on a careful analysis of the conclusions from the modern trends in \glspl{uav} outlined in \cref{ch:modern_trends} and the objectives reviewed in \cref{ch:objectives}, the following requirements are established for the high-level system as well as the detailed requirements for each of the components of the system: the \gls{uav}, the control station, and the reconnaissance platform.

\section{High-level System Requirements}

The high-level system requirements are as follows:

\begin{itemize}
  \item The system must be able to operate in remote areas with limited infrastructure, such as roads, electricity, and internet connectivity.

  \item The system must be able to be monitored remotely, with the ability to communicate with a ground station in real-time and for long distances, more than \SI{5}{\kilo\meter}.

  \item The system must be cost-effective, with the capability to be assembled and disassembled easily, and to be repaired and maintained with minimal effort.

  \item The system must be modular, allowing for the integration of different sensors and payloads for different applications, such as mapping, surveillance, and monitoring the environment.

  \item The system must be able to perform reconnaissance tasks autonomously, with the ability to take off, land, and navigate given a set of waypoints. Moreover, the system must be able to update its flight plan in real-time.

  \item The system must integrate a real use-case scenario to proof the concept of the system, with the ability to detect and track objects, monitor the environment, and generate alerts and notifications in case of critical events or failures.

  \item The system must comply with the applicable regulatory framework for \glspl{uav} in the country of operation, Spain, as well as the \gls{eu} regulations. See \cref{ch:regulatory_framework} for more information.
\end{itemize}

Given the high-level system requirements, the detailed requirements for each of the components of the system are outlined in the following sections.

\section{Unmanned Aerial Vehicle Requirements}

For the \gls{uav}, the specifications necessary to meet the high-level system requirements are:

\begin{itemize}
  \item The \gls{uav} must be able to be controlled remotely, with the ability to communicate with a ground station in real-time and for long distances, more than \SI{5}{\kilo\meter} and function autonomously.

  \item The \gls{uav} must be able to carry different payloads and sensors for different applications up to a maximum payload weight of \SI{2}{\kilogram}, with the ability to adapt to different reconnaissance tasks.

  \item The \gls{uav} must be able to fly for a minimum of \SI{30}{\minute}, without the need for recharging.

  \item The \gls{uav} must be have a failsafe mechanism, that is it must be able to return to the control station in case of loss of communication or other critical failures.

  \item The \gls{uav} must be able to process data in real-time, with the ability to relay the information to the control station and the reconnaissance platform.

  \item The \gls{uav} must comply with the EASA regulations for the Open Category, with a maximum limit set at \SI{25}{\kilogram} of MTOW and \SI{3}{\meter} of wingspan.
\end{itemize}

\section{Control Station Requirements}

In order to provide an autonomous and reliable control of the \gls{uav}, the control station requirements are as follows:

\begin{itemize}
  \item The control station must be able to receive telemetry data from the \gls{uav} in real-time, with the ability to send commands to the \gls{uav} to update its flight plan.

  \item The control station must be able to be used remotely, with the ability to communicate with the \gls{uav} over long distances, more than \SI{5}{\kilo\meter}, and for extended periods of time.

  \item The control station must be able to create a geofence around the area of operation, with the ability to monitor the \gls{uav}'s position and altitude in real-time and receive alerts and notifications in case of critical events or failures.

  \item The control station must log all telemetry data and flight information, with the ability to analyze the data and generate reports.

  \item The system used for the control station must be reliable, secure, and easy to use, with the addition of a backup control station in case of failure.
\end{itemize}

\section{Reconnaissance Platform Requirements}

Finally, to provide a use-case scenario for the system, the reconnaissance platform requirements are as follows:

\begin{itemize}
  \item The reconnaissance platform must be able to run on a variety of operating systems, with the ability to communicate with the \glspl{uav} and the control station in real-time and over long distances.

  \item The reconnaissance platform must be customizable, allowing for the integration of new features and the modification of existing ones, as well as, the addition of new \glspl{uav} to the system and different types of reconnaissance tasks.

  \item The reconnaissance platform must have alerting and notification capabilities, with the ability to send alerts and notifications to the user in case of critical events or failures.

  \item The reconnaissance platform must have a user-friendly interface, with the ability to display telemetry data and flight information in real-time, as well as, the ability to monitor the \glspl{uav} in real-time.
\end{itemize}

% Local Variables:
% jinx-local-words: "uav"
% End:

\chapter{Design}\label{ch:design}

\todo{write this chapter}

\chapter{Implementation}
\label{ch:implementation}

\todo{write this chapter}


\oldpart{Results}\label{part:results}

\chapter{Testing}\label{ch:testing}

In this chapter, the different parts of the systems have been tested following the V-model \cref{ch:methodology_approach}. In the following sections, the different tests are described and the results are presented.

\section{Individual Testing}\label{sec:individual_testing}

In this section, the components are tested individually to ensure that they are working as expected, that is the drone, the reconnaissance platform, and the communication system.

\subsection{Drone Testing}\label{subsec:drone_testing}

In order to test the drone, first static tests where taken and all the electronics was tested individually. For the motors, the ESCs and the flight controller, the first thing is to calibrate them and adjust all ports. This is done through the ArduPilot program \autocite{ardupilotMissionPlanner}. After the configuration is done, the tests are done without propellers as it can be a safty hazard to do the tests with them on, refer to \cref{fig:drone_testing}. Also it is worth noting than whenever the drone is being manipulated, the battery must be disconnected to avoid any unexpected situations.

\begin{figure}
	\includegraphics{testing_drone.jpeg}
	\caption{Drone testing without propellers}\label{fig:drone_testing}
\end{figure}

Once the drone is fully working statically, then the next step is to configure all the sensor, \gls{gps} and \gls{imu} in this case. Once all the sensors are configured, the next step is to test them individually. The \gls{gps} is tested by checking the position in the ArduPilot program and the \gls{imu} is tested by checking the orientation in the \gls{imu} program.

\subsection{Reconnaissance Platform Testing}\label{subsec:reconnaissance_platform_testing}

For the reconnaissance platform, the first thing to do is that everything works individually. The camera is tested by connecting it to the computer and checking that the camera is working. The camera is tested by taking a picture and checking that the picture is saved in the computer. The camera is also tested by checking that the camera is working with the Jetson Nano connected, see \cref{fig:reconnaissance_platform_testing}.

\begin{figure}
	\hfill
	\begin{subfigure}{0.4\textwidth}
		\includegraphics{testing_jetson.jpeg}
		\caption{Installation and configuration of the Jetson Nano.}\label{fig:jetson_testing}
	\end{subfigure}
	\hfill
	\begin{subfigure}{0.4\textwidth}
		\includegraphics{testing_camera_jetson.jpeg}
		\caption{Testing the camera with the Jetson Nano.}\label{fig:jetson_camera_testing}
	\end{subfigure}
	\hfill

	\caption{Testing of the reconnaissance platform.}\label{fig:reconnaissance_platform_testing}
\end{figure}

Moreover, the server developed for the reconnaissance platform is tested by connecting the camera to the Jetson Nano and checking that the server is working. The server is tested by checking if the system is able to recognize the different objects in the image and if the system is able to send the information to the computer.

\subsection{Communication System Testing}\label{subsec:communication_system_testing}

For the communication system, the tests are done by connecting the \gls{rf} module and the \gls{4g} module to the computer and checking that the modules are working. The \gls{rf} module is tested by checking that the module is working with the computer and the \gls{4g} module is tested by checking that the module is working with the computer.

\section{General Testing}\label{sec:general_testing}

Once all the individual tests are done, the next step is to test the system as a whole. The system is tested by connecting all the components and checking that the system is working. The system is tested by checking that the drone is able to fly and that the reconnaissance platform is able to recognize the different objects in the image. The system is also tested by checking that the communication system is working and that the system is able to send the information to the computer.

For this, the drone was flown in a controlled environment in a authorized zone to comply with the regulations, see \cref{ch:regulatory_framework}. The tests were performed to test if the system was able to recongize human beings in an area of \SI{10}{\kilo\metre^2}. Moreover, all videos and images were stored in the computer and the system was able to send the information to the computer. In \cref{fig:detection_human}, an example of the system detecting a human being is shown. And also in \cref{fig:llm_detection}, the whole system working is shown and the end result of detecting and processing the images is performed.

\begin{figure}
	\includegraphics{detection_human.jpg}
	\caption{Human detection during the tests}\label{fig:detection_human}
\end{figure}


\chapter{Results}\label{ch:results}

Finally after all the tests were done, the system was validated, and fully working, the results of the systems obtained are:

\begin{itemize}
	\item The drone is able to fly autonomously and detect objects in the environment.
	\item The drone is able to communicated and send information to a server to receive real-time information.
	\item The drone is able to have a 30 to 40 minutes of flight time.
	\item Integration with \glspl{llm} make the system able to adapt to detect different characteristics of objects.
	\item The drone is modular and can be easily repaired and maintained. Moreover, the drone is able to be easily modified to adapt to different applications.
\end{itemize}

With all this results obtained, the system checks all the requirements and objectives of the project, refer to \cref{ch:requirements} , and the system is able to be used in different applications, such as surveillance, search and rescue, and monitoring of different areas.

For more videos demonstrating the system working, refer to the following links: \url{https://youtu.be/F1fDpXw-kBg}, \url{https://youtu.be/kdMgwRRte-8}, and \url{https://youtu.be/IqNsVOT3uFQ}.

\section{Problems Encountered}\label{sec:problems_encountered}

As the project is a proof of concept, there were some problems encountered during the development of the system. The main problems encountered were maily due to the nature of the project, as this project is a proof of concept, there were not previous projects to compare with, and the development of the system was done from scratch.

The first problem encountered was the design and the selection of the components. In order to create a system that is able to detect objects in the environment and more importantly, to be able to have enough flight duration, the selection of the components was crucial. The selection of the components was done based on the requirements of the project, and the components were selected based on the cost, the availability, and the compatibility with the system. One problem was that at first a wrong type of \gls{esc} was chosen and it was not compatible to the current set up. And in the end, the correct type was selected, the Tmotor FLAME 100A LV 600Hz, refer to \cref{subsec:design_propulsion_system}.

Another problem encountered was the placement of the components in the \gls{uav}. As the \gls{uav} has limitted space, the components had to be carfully place in order to have space and be stable during flight.

Finally, multiple problems were confronted recarding the integration of all systems and the reconnaissance platform. Here, the libraries used to capture the camera live-feed had to be modified in order to be compatible to the Ricoh Theta X camera. Moreover, custom communication protocols and servers had to be developed in order to be able to send the information from the drone to the server and process the information in real-time.

All in all, after multiple tests and iterations, the system was able to be fully working and the problems encountered were solved.


\oldpart{Conclusions}\label{part:conclusions}

\chapter{Conclusions}\label{ch:conclusions}

This thesis presents a comprehensive approach to developing and implementing an autonomous drone system for \gls{ntn} in remote areas. The project successfully achieved its objectives of creating a modular, cost-effective, and open-source solution that can be easily adapted for various reconnaissance tasks. The implemented system demonstrates the potential of integrating \glspl{uav} with advanced communication technologies and artificial intelligence to enhance connectivity and data collection in challenging environments.

The designed \gls{uav} platform, equipped with a 360-degree camera and onboard processing capabilities, proved capable of autonomous flight and real-time object detection. The integration of a secure communication system, combining \gls{rf} and \gls{4g} technologies, enabled reliable data transmission between the drone and the control station. Furthermore, the development of a sophisticated reconnaissance platform, leveraging deep learning algorithms and \glspl{llm}, showcased the system's ability to provide valuable insights from collected data.

The project's modular approach and use of commercially available components make it accessible for other researchers and organizations to replicate and build upon. This aligns well with the goal of fostering new research in the field of non-terrestrial networks and low-altitude drone-based systems. The successful implementation and testing of the system in a controlled environment demonstrated its potential for applications in humanitarian aid, environmental monitoring, and disaster response.

While some challenges were encountered during the development process, particularly in component selection and system integration, these were ultimately overcome through iterative design and testing. The final system met all specified requirements, including flight duration, autonomous operation, and real-time data processing capabilities.

In conclusion, this thesis contributes a valuable proof-of-concept for autonomous drone systems in non-terrestrial networks, paving the way for future advancements in this field. The developed platform offers a flexible foundation for further research and practical applications in remote area connectivity and reconnaissance.

\chapter{Future works}\label{ch:future_work}

\todo{write this chapter}


\blankpage%
\printbibliography%

\end{document}
