\section{Communication System}\label{sec:design_communication_system}

The communication system is the system that allows the \gls{uav} to communicate with the control station in real-time, as well as providing to the reconnaissance platform. Different communication systems can be used depending on the use case, such as \gls{5g}, \gls{4g}, \gls{3g}, \gls{wifi}, or \gls{rf}, and the environment where the \gls{uav} will operate. The characteristics to consider when choosing a communication system are shown in \cref{tab:communication_system_characteristics}.

\begin{table}
  \begin{tabular}{ c c c c c }
    \toprule
    \textbf{Communication System} & \textbf{Reliability} & \textbf{Latency} & \textbf{Range} & \textbf{Bandwidth} \\
    \midrule
    \gls{5g} & High & Low & Long & Really High \\
    \gls{4g} & High & Low & Long & High \\
    \gls{3g} & Medium & Medium & Medium & Medium \\
    \gls{wifi} & Low & High & Short & High \\
    \gls{rf} & High & Low & Long & Medium \\
    \bottomrule
  \end{tabular}
  \caption{Comparison of the different communication systems depending on their characteristics.}\label{tab:communication_system_characteristics}
\end{table}

The communication system for the drone is divided into two main subsystems: the communication systems in charge of connecting the \gls{uav} with the control station, and the communication systems in charge of connecting the on-board reconnaissance platform with the off-board reconnaissance platform.

For the communication system in charge of connecting the \gls{uav} with the control station, a \gls{rf} communication module was chosen, as it provides a stable and reliable connection, as well as a low latency and long range. Moreover, it is the defacto standard for \gls{uav} communication systems to communicate with the control station in real-time. The module chosen to connect the \gls{uav} with the control station is the RFD868 TXMOD V2 868Mhz 1W \autocite{rcinnovationsComprarMdulos}, as it is the defacto standard for long-range \gls{rf} communication systems for \glspl{uav} and it is compatible with the \gls{eu} regulations for \gls{rf} communication systems, refer to \cref{subsec:regulation_eu_radio_equipment_directive}, and it provides the highest range as it uses the \SI{868}{\mega\hertz} frequency band.

Likewise, for the communication system, a \gls{4g} communication module was chosen as it provides a high bandwidth to send the data collected by the \gls{uav} in real-time. The router chosen is the \todo{add ref to where we bought them} due to a good price-performance ratio and a low weight and size. Note that a \gls{5g} router can also be used but due to the higher cost and the lack of coverage in the area of operation, a \gls{4g} router was chosen instead.

% Local Variables:
% jinx-local-words: "defacto rcinnovationsComprarMdulos rf uav"
% End:
