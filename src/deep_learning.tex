\chapter{Deep Learning}\label{ch:deep_learning}

\gls{dl} is a subfield of \gls{ml} that focuses on the development of algorithms and models inspired by the structure and function of the human brain. These algorithms are designed to learn from data, identify patterns and relationships, and make predictions or decisions without explicit instructions.\ \gls{dl} algorithms are characterized by their ability to automatically discover and extract features from raw data, enabling them to perform complex tasks such as image recognition, speech recognition, and natural language processing.

\gls{dl} has revolutionized various industries and domains, including healthcare, finance, transportation, and entertainment. By leveraging the power of \gls{dl}, organizations can analyze large datasets, extract valuable insights, and automate complex tasks, leading to improved decision-making, enhanced user experiences, and optimized processes. From self-driving cars and virtual assistants to medical diagnostics and fraud detection, \gls{dl} is transforming the way we interact with technology and the world around us.

Furthermore, \gls{dl} plays a crucial role in enabling \gls{uav} autonomy, allowing drones to perform tasks such as navigation, obstacle avoidance, and object recognition without human intervention. By integrating \gls{dl} algorithms into \gls{uav} systems, researchers and developers can enhance the capabilities and efficiency of drones, enabling them to operate in complex environments and execute sophisticated missions.

\section{Deep Learning Techniques}

\gls{dl} encompasses a wide range of techniques and architectures that enable machines to learn from data and make decisions. Some of the most common \gls{dl} techniques include:

\begin{itemize}
  \item \textbf{\glspl{ann}:} \glspl{ann} are computational models inspired by the structure and function of the human brain. They consist of interconnected nodes, or neurons, organized in layers, with each neuron performing a simple computation.\ \gls{ann} can learn complex patterns and relationships in data through a process called back-propagation, where errors are propagated back through the network to adjust the model's parameters.\ \gls{ann} are used in a variety of tasks, such as classification, regression, and clustering.

  \item \textbf{\glspl{cnn}:} \glspl{cnn} are a type of \glspl{ann} designed for processing and analyzing visual data, such as images and videos. They use convolutional layers to extract features from input data, pooling layers to reduce spatial dimensions, and fully connected layers to make predictions.\ \gls{cnn} are widely used in image recognition, object detection, and image segmentation tasks.

  \item \textbf{\glspl{rnn}:} \glspl{rnn} are a type of \glspl{ann} designed for processing sequential data, such as time series, text, and speech. They have feedback connections that allow information to persist over time, enabling them to capture temporal dependencies in data.\ \gls{rnn} are used in natural language processing, speech recognition, and machine translation tasks.

  \item \textbf{\glspl{gan}:} \glspl{gan} are a type of a \gls{dl} model that consists of two neural networks, a generator and a discriminator, trained adversarially. The generator generates synthetic data samples, while the discriminator distinguishes between real and fake samples.\ \gls{gan} are used in image generation, style transfer, and data augmentation tasks.
\end{itemize}

These techniques form the foundation of \gls{dl} and are used in a wide range of applications across various domains, enabling machines to perform complex tasks and make intelligent decisions. In the context of \gls{ntn} and \glspl{uav}, \gls{dl} techniques can enhance network performance, optimize resource allocation, and enable autonomous operation, leading to more efficient and reliable systems. Furthermore, \gls{dl} can enable \glspl{uav} to perform tasks such as navigation, object detection, and mission planning with high accuracy and efficiency, making them valuable tools for a wide range of applications (e.g., surveillance, monitoring, and disaster response).
