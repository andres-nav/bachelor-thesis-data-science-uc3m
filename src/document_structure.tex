\chapter{Document Structure}\label{ch:document_structure}

The document structure of this thesis is organized into six main parts, each focusing on a specific aspect of the research on autonomous drones for Non-Terrestrial Networks (NTNs) in remote areas.

The Theoretical Background part, \cref{part:theoretical_background}, provides the foundational knowledge necessary to understand the core concepts of the thesis. It delves into non-terrestrial networks, explaining different types such as geostationary satellites, low-earth orbit satellites, and high-altitude platforms. This part also covers unmanned aerial vehicles, their types, and applications, as well as an overview of deep learning techniques relevant to the project.

The State of the Art part, \cref{part:state_of_the_art}, explores the historical development of non-terrestrial networks and UAVs, current types and characteristics of NTN platforms, and modern trends in the field. This part provides context for the research and highlights the gaps in existing solutions that this thesis aims to address.

The Methodology part, \cref{part:methodology}, forms the core of the thesis, detailing the requirements, design, and implementation of the proposed system. It covers the unmanned aerial vehicle design, control station setup, reconnaissance platform development, and communication system integration. This part provides a comprehensive overview of how the system was built and configured.

The Results part, \cref{part:results}, presents the outcomes of the project, including the testing procedures for individual components and the system as a whole. It discusses the achievements of the implemented system, as well as the challenges encountered during development and how they were overcome.

Finally, the Conclusions part, \cref{part:conclusions}, summarizes the key findings of the research, reflects on the objectives achieved, and proposes directions for future work. It also includes a discussion on the socio-economic impact of the project and an analysis of the relevant regulatory framework governing UAV operations.
