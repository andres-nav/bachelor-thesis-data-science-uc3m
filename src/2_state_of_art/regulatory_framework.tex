\chapter{Regulatory Framework}
\label{ch:regulatory_framework}

The regulatory framework governing drones is a complex and dynamic area, influenced by various laws and regulations that differ from country to country. Generally, drone operations are regulated by aviation authorities responsible for ensuring safe and responsible usage.

\section{Relevant Institutions}

\subsection{European Union Aviation Safety Agency (EASA)}
The European Union Aviation Safety Agency (EASA) \autocite{eu-1139-2018} plays a crucial role in harmonizing aviation safety standards across all EU member states. Its primary objective is to maintain a consistent and high level of safety in civil aviation operations throughout the European Union. EASA achieves this through the establishment and enforcement of common regulations applicable to all member states. Notably, for the standardization of Unmanned Aerial Systems (UAS), EASA has implemented Regulations (EU) 2019/947 \autocite{eu-947-2019} and (EU) 2019/945 \autocite{eu-945-2019}.

\subsection{Spanish Aviation Safety and Security Agency (AESA)}
In Spain, the Spanish Aviation Safety and Security Agency (AESA) \autocite{sp-184-2008} serves as the national regulatory authority, overseeing compliance with civil aviation standards within the aerospace sector. AESA plays a critical role in promoting the development and application of aviation legislation, ensuring that the Spanish civil aviation system upholds the highest safety, quality, and sustainability standards. In instances of non-compliance with aviation regulations, AESA possesses the authority to enforce sanctions.

\section{Applicable Legislation}

\subsection{Implementing Regulation (EU) 2019/947}
The Implementing Regulation (EU) 2019/947 \autocite{eu-947-2019} establishes the operational rules and requirements for UAS within the European Union. It provides a legal framework for the utilization of UAS across various operational categories, outlining requirements for operational authorizations and risk assessments where applicable. The regulation sets standards for remote pilot competency, operational procedures, and safety management to conduct UAS flights safely and effectively.

Additionally, it integrates with the Delegated Regulation (EU) 2019/945 \autocite{eu-945-2019} by defining operational requirements related to the UAS classes established within it. The regulation details specific operational limitations and conditions for each UAS class, including the management of UAS in classes C0 through C4. It also includes provisions for the safe integration of newly introduced UAS classes under Delegated Regulation (EU) 2020/1058 \autocite{eu-1058-2020}, specifically classes C5 and C6.

Moreover, this regulation addresses the procedures for UAS operators from third countries (non-EASA member states) wishing to operate within the Single European Sky (SES) airspace, ensuring alignment with EU standards and safety regulations.

\subsection{Delegated Regulation (EU) 2019/945}
The Delegated Regulation (EU) 2019/945 \autocite{eu-945-2019} defines the rules and standards for UAS within the European Union. It specifies the types of UAS that require certification regarding design, production, and maintenance. This regulation also provides guidelines for the commercialization of UAS intended for use in the Open category, as well as for remote identification accessories (e.g., Drone Remote ID). Furthermore, it outlines the requirements for the design and manufacture of UAS intended for operations defined in the Implementing Regulation (EU) 2019/947.

\subsection{Regulation (EU) 2024/1689: Artificial Intelligence Act}
The Artificial Intelligence Act (AI Act) of the European Union \autocite{AIActIntoForce}, which came into force on the 1st of August 2024, aims to ensure that AI systems are safe, transparent, and ethical, while fostering innovation and protecting fundamental rights as stated in the Delegated Regulation (EU) 2024/1689 \autocite{eu-1689-2024}. The Act categorizes AI systems by risk, imposing strict requirements on high-risk applications, particularly in aviation, which may affect public safety and fundamental rights. These requirements encompass robust risk management, transparency, human oversight, and data governance, ensuring that AI systems are reliable and secure.

The AI Act introduces significant compliance obligations that could escalate development costs and timelines. High-risk systems must adhere to stringent standards to access the EU market, potentially challenging innovation but ultimately aiming to build trust and facilitate broader adoption of AI technologies within the EU.

\section{Operational Categories}
The Regulation (EU) 2019/947 \autocite{eu-947-2019} classifies UAS into three distinct operational categories:

\begin{itemize}
  \item \textbf{Open Category:} The least restrictive category, designed for low-risk operations, includes activities such as recreational flying and commercial operations posing minimal risk to people and property. Operators must adhere to specific limitations (e.g., flying below 120 meters, maintaining Visual Line of Sight). UAS must weigh under 25 kg, and pilots must ensure that the drone does not fly over people or in restricted areas. No prior authorization is required, though registration and remote pilot training are compulsory for all operations, except for drones weighing less than 250 g that lack a camera or sensor.

  \item \textbf{Specific Category:} This category covers medium-risk operations necessitating a more detailed assessment. It includes operations that may involve flying over people or in restricted areas, provided mitigation procedures are in place. Operators must conduct a risk assessment and obtain an operational authorization known as Standard Training Scenarios (STS) from AESA. Requirements for UAS and pilot qualifications may vary based on the specific risk assessment and operational procedures defined within it.

  \item \textbf{Certified Category:} Designed for high-risk operations, this category involves stringent requirements comparable to those for manned aviation. UAS must meet specific certification standards, and operators must comply with strict safety regulations. This category often includes advanced training requirements and operational procedures similar to those for commercial air transport.
\end{itemize}

\subsection{Open Category}
This work will focus on civil UAS that fall under EASA's Open Category, although some findings may be applicable to other categories with appropriate regulatory adjustments. Within the Open Category, three subcategories differentiate based on associated risk, aircraft weight, and operational limits:

\begin{enumerate}
  \item \textbf{A1}: UAS with a Maximum Takeoff Weight (MTOW) of less than 250 g that can fly over people but not over assemblies of people.

  \item \textbf{A2}: UAS with an MTOW of less than 4 kg that can fly close to people but must maintain a horizontal distance of 30 meters (5 meters in low-speed configuration).

  \item \textbf{A3}: UAS with an MTOW of less than 25 kg that must maintain a horizontal distance of 150 meters from residential, commercial, industrial, or recreational areas.
\end{enumerate}

\begin{figure}
  \includegraphics{eu_regulations_open_category_chart.png}
  \label{fig:eu_regulations_open_category_chart}
  \caption{EU Regulations Open Category chart describing the subcategories A1, A2, and A3 with their respective operational limitations extracted from \autocite{ageagleEuropeanUnion}}
\end{figure}

Additional rules applicable to all three subcategories include:

\begin{itemize}
  \item The maximum height must not exceed 120 meters above ground level, as the lower limit for general aviation is 150 meters. This leaves only a 30-meter separation between manned aviation and UAS.

  \item Operators must always maintain Visual Line of Sight (VLOS) unless the aircraft is in ``follow me'' mode or the pilot is using First-Person View (FPV) goggles.

  \item Operators must register if the UAS weighs more than 250 g or if the aircraft is equipped with a camera or sensor.

  \item The aircraft must possess a remote identification ID, which is standard in all C1-C6 categories, with the exception of C4 and privately built aircraft.
\end{itemize}
