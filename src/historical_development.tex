\chapter{Historical Development}\label{ch:historical_development}

The evolution of \glspl{ntn} and \glspl{uav} has been shaped by technological advances and the increasing demand for global connectivity over the past several decades. Originally, \glspl{ntn}, encompassing satellite communication networks, \glspl{hap}, and \glspl{uav}, were developed for specialized applications. These early systems were primarily used for military, navigation, television broadcasting, remote sensing, and disaster management purposes. Due to the high costs and complexities associated with the manufacturing, launching, and maintaining these systems, their deployment was limited to specific sectors and regions, often focusing on government or large corporate projects.

Early satellite communication networks were dominated by \gls{geo} satellites, which provided consistent coverage over specific areas of the Earth, particularly for television broadcasting and weather forecasting. However, the high latency and large round-trip time associated with \gls{geo} satellites, positioned at approximately \SI{36000}{\kilo\meter} from Earth, posed challenges for expanding their use to real-time communication services. Additionally, the prohibitive costs and challenges of deploying and maintaining \gls{geo} satellites restricted their usage largely to commercial and government-backed projects.

Throughout the late 20th century, \glspl{ntn} remained niche solutions, but technological advancements and the growing need for more comprehensive and reliable global connectivity shifted the focus. The limitations of \glspl{tn}, particularly in rural, remote, and inaccessible regions such as deserts, oceans, and mountainous areas, drove the demand for new approaches. Expanding terrestrial network coverage into these regions posed economic and logistical challenges, making \glspl{ntn} a critical complementary solution. Satellites became vital to extending coverage beyond the reach of terrestrial infrastructure, filling gaps where ground-based systems were either impractical or uneconomical to deploy.

The 1990s saw the rise of \gls{leo} satellite constellations, which were developed to overcome some of the inherent limitations of \gls{geo} satellites.\ \gls{leo} satellites, operating at much lower altitudes of \SI{300}{\kilo\meter} to \SI{1500}{\kilo\meter}, offered significantly reduced latency and improved spectral efficiency. These benefits made \gls{leo} satellites more suitable for supporting new and emerging applications that demanded real-time communication and data transmission. Despite the technological promise of \gls{leo} satellites, the initial wave of mega-constellation projects—large networks comprising hundreds to thousands of satellites—stalled, largely due to the high costs of deployment and a lack of sustainable business models.

The late 1990s and early 2000s marked a renewed interest in integrating \glspl{ntn} with terrestrial systems, particularly as global internet access became a key societal goal. As mobile network generations progressed from \gls{2g} to \gls{4g}, the need for more adaptive network solutions grew. However, it was not until the development of \gls{5g}, spearheaded by the \gls{3gpp}, that serious efforts were made to fully integrate \glspl{ntn} with terrestrial networks.\ \gls{3gpp}'s Release 15 \autocite{3gpp_rel15} in 2018 laid the foundation for \gls{5g} networks, and subsequent releases aimed to include \glspl{ntn}, such as satellite and \glspl{hap}, as essential components of the \gls{5g} ecosystem. These efforts recognized \glspl{ntn}' potential to expand the reach of \gls{5g} networks into underserved areas and enhance service reliability in mobile broadband and \gls{iot} applications.

During the same period, advancements in \glspl{uav} also contributed to the development of \glspl{ntn}. Initially developed for military and surveillance applications, \glspl{uav} began to be explored for their potential in civil applications such as disaster management, agriculture, and communications. The rise of \gls{5g} networks allowed for \glspl{uav} to be integrated into terrestrial networks, enabling beyond \gls{vlos} operations that required low-latency, reliable connections for autonomous vehicles, precision agriculture, and more.

As \gls{5g} networks continue to evolve, \glspl{ntn} are becoming increasingly vital in ensuring seamless global connectivity.\ \gls{leo} constellations, in particular, have seen a resurgence, with companies like SpaceX (Starlink) \autocite{tao2022impact} and OneWeb \autocite{zhu2022laser} developing large satellite networks to deliver low-latency, high-speed internet services to remote areas. These \glspl{ntn} are providing the much-needed infrastructure to bridge the digital divide by offering global coverage, enhancing reliability, and addressing specific issues such as network scalability and latency that have traditionally limited satellite communications.

% Local Variables:
% jinx-local-words: "iot ntn uav"
% End:
