\chapter{Regulatory framework}

\TODO{rewrite the whole chapter}
The regulatory framework for drones is a complex and evolving field. The use of drones is regulated by a variety of laws and regulations, which vary from country to country. In general, the use of drones is regulated by aviation authorities, which are responsible for ensuring that drones are operated safely and responsibly.

\section{Relevant Institutions}

\subsection{European Union Aviation Safety Agency (EASA)}

The European Union Aviation Safety Agency (EASA) is the authority responsible for harmonizing aviation safety standards across all EU member states. EASA’s primary objective is to ensure a consistent and high level of safety in civil aviation operations throughout the European Union. This is achieved through the establishment and enforcement of common regulations applicable to all member states. In particular, for the standardization of U AS, EASA has implemented Regulations (EU) 2019/947 and (EU) 2019/945.

\subsection{Spanish Aviation Safety and Security Agency (AESA)}

The Spanish Aviation Safety and Security Agency (AESA) serves as the national regulatory body in Spain and is tasked with overseeing adherence to civil aviation standards within the country’s aerospace sector. AESA plays a critical role in promoting the development and application of aviation legislation, ensuring that the Spanish civil aviation system upholds the highest standards of safety, quality, and sustainability. In cases of non-compliance with aviation regulations within Spain, AESA has the authority to enforce sanctions.

\section{Applicable Legislation}

\subsection{Implementing Regulation (EU) 2019/947}

The Implementing Regulation (EU) 2019/947 is an European Union regulation that establishes the operational rules and requirements for (U AS). It provides a legal framework for the use of U AS across different categories of operations. The regulation outlines the operational requirements and procedures for U AS operators, including the need for operational authorizations and risk assessments when applicable. It sets standards for remote pilot competency, operational procedures, and safety management necessary to conduct U AS flights safely and effectively.

Additionally, the Implementing Regulation (EU) 2019/947 integrates with the Delegated Regulation (EU) 2019/945 by defining operational requirements in relation to the U AS classes established within. It details the specific operational limitations and conditions for each class of U AS, including requirements for the handling of U AS in classes C0 through C4, and includes provisions for the safe integration of new U AS classes introduced by the amendment in Delegated Regulation (EU) 2020/1058, classes C5 and C6.

This regulation also addresses the procedures for U AS operators from third countries (non-EASA member states) seeking to operate within the Single European Sky (SES) airspace, ensuring that their operations align with EU standards and safety regulations.

\subsection{Delegated Regulation (EU) 2019/945}

The Delegated Regulation (EU) 2019/945 is an European Union regulation that sets the rules and standards for U AS. It defines the types of U AS that require certification in terms of design, production, and maintenance. The regulation also establishes guidelines for the commercialization of U AS intended for use in the Open category, as well as for remote identification accessories (e.g. Drone Remote ID).

It also outlines the requirements for the design and manufacture of U AS intended for use under the conditions defined in the Implementing Regulation (EU) 2019/947.

\subsection{Royal Decree UAS 517/2024}

Its purpose is to establish the legal framework for those issues where the Implementing and Delegated Regulations from EASA either grant member states (e.g., Spain) the authority to regulate or do not directly address these aspects.

\subsection{Regulation (EU) 2024/1689: Artificial Intelligence Act}

The Artificial Intelligence Act (AI Act) of the European Union entered into force on the 1st of August 2024 \autocite{AIActIntoForce} and aims to ensure AI systems are safe, transparent, and ethical while fostering innovation and protecting fundamental rights \autocite{eu-1689-2024}. It categorizes AI systems by risk, imposing strict requirements on high-risk applications, such as those in aviation, which may impact public safety and fundamental rights. These requirements include robust risk management, transparency, human oversight, and data governance, ensuring AI systems are reliable and secure.

The AI Act introduces significant compliance obligations that could increase development costs and time. High-risk systems must meet strong standards to access the EU market, which may challenge innovation but ultimately aims to build trust and facilitate the broader adoption of AI technologies within the EU.

\section{Operational Categories}

Regarding U AS, the Implementing Regulation (EU) 2019/947 defines three distinct
categories \autocite{eu-947-2019}:

\begin{itemize}
    \item \textbf{Open Category:} This is the least restrictive category and is designed for low-risk operations. It includes activities such as recreational flying and commercial operations that pose minimal risk to people and property. Operators must follow specific operational limitations (e.g., flying below 120 meters, maintaining V LOS). U AS must be under 25 kg, and the pilot must ensure the drone does not fly over people or in restricted areas. No prior authorization is required, but registration and training as a remote pilot are compulsory for all operations, with the exception of drones weighing <250 g that do not fit a camera/sensor.

    \item \textbf{Specific Category:} This category covers medium-risk operations where a more detailed assessment is needed. It includes operations that might involve flying over people or in restricted areas but with mitigation procedures in place. Operators must conduct a risk assessment and obtain an operational authorization called Standard Training Scenarios (ST S) from AESA. The requirements for U AS and pilot qualifications can vary based on the specific risk assessment and operational procedures defined in the risk assessment.

    \item \textbf{Certified Category:} This category is designed for high-risk operations that are more complex and involve significant risk to people and property, such as those similar to manned aviation operations. It involves stringent requirements similar to those for manned aviation. UAS must meet specific certification standards, and operators need to comply with strict safety regulations. It often includes requirements for advanced training and operational procedures, similar to those for commercial air transport.
\end{itemize}

\subsection{Open Category}

This work will focus on civil U AS that fall under EASA\textsf\textit{'}s Open Category, although
some of the work done may be applicable to other categories with the proper regulatory adjustments.
Within the Open Category, there are three subcategories that differentiate themselves based on the associated risk, aircraft weight, and operational limits:

\begin{enumerate}
  \item A1: UAS with an MTOW of less than 250 g that can fly over people but not over assemblies of people.

  \item A2: UAS with an MTOW of less than 4 kg that can fly close to people, but must maintain a horizontal distance of 30 meters (5 meters in low-speed configuration).

  \item A3: UAS with an MTOW of less than 25 kg that must maintain a horizontal distance of 150 meters from residential, commercial, industrial, or recreational areas.
\end{enumerate}

\todo{add figure with the different categories}

Some additional rules that apply to all three subcategories are stated below:

\begin{itemize}
  \item The height of 120 meters above ground level should not be exceeded, as the lower limit for general aviation is 150 meters. Therefore, there is only a 30-meter separation between manned aviation and UAS.

  \item The operator must always fly in Visual Line Of Sight (VLOS), unless the aircraft is in \textit{follow me} mode or the pilot is using First-Person View (FPV) goggles.

  \item The operator must be registered if the UAS weighs more than 250 g or if the aircraft is equipped with a camera or sensor.

  \item The aircraft must have a remote identification ID\autocite{eu-945-2019}, which comes by default in all C1-C6 categories with the exception of C4 and privately built aircraft.
\end{itemize}
