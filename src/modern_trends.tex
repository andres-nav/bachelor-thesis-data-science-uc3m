\chapter{Modern Trends}\label{ch:modern_trends}

In recent years, \glspl{ntn} have emerged as a significant advancement in the field of wireless communication. These networks utilize platforms as described in previous chapters (e.g., satellites, \glspl{hap}, or \glspl{uav}), to provide connectivity in areas where traditional communication infrastructure may be lacking or underdeveloped. As the demand for reliable communication continues to grow, \glspl{ntn} present unique opportunities for various applications, particularly in disaster response, virtual reality, and high-speed internet access. This section explores the modern trends of \glspl{ntn}, highlighting their importance in enhancing \gls{uav} capabilities and addressing contemporary challenges.

One of the most impactful applications of \gls{ntn}-enabled \glspl{uav} is in rescue operations during natural disasters. These drones can serve as critical tools in emergencies such as earthquakes, hurricanes, and floods by establishing essential communication links between affected areas and external agencies. For instance, during the 2015 Nepal earthquake \autocite{drone_nepal_2015}, \glspl{uav} equipped with \gls{ntn} technology were deployed to transmit real-time video feeds of devastated regions, allowing emergency responders to assess damage and prioritize resources effectively. Additionally, \glspl{uav} can be outfitted with thermal imaging cameras to locate survivors in disaster zones. In the aftermath of Hurricane Harvey in 2017 \autocite{Greenwood2020FlyingIT}, \gls{ntn}-enabled \glspl{uav} scanned flooded neighborhoods, relaying findings to command centers and guiding rescue teams to those in need. The integration of \glspl{ntn} with \glspl{uav} thus creates a network of aerial assets that optimizes response times and enhances overall disaster management efforts.

Beyond rescue operations, \gls{ntn}-enabled \glspl{uav} are being utilized in innovative applications such as \gls{vr} experiences and \gls{5g} connectivity. By transmitting high-definition video feeds, these drones offer immersive experiences that allow users to engage with their surroundings in unique ways. For instance, during training scenarios for emergency responders, \glspl{uav} can capture 360-degree videos of disaster zones, providing trainees with a realistic simulation of on-ground conditions \autocite{vr_model_drone_footage_disaster_zone}. This not only enhances training effectiveness but also helps responders make informed decisions based on a comprehensive understanding of the situation. Moreover, companies like Nature Eye \autocite{natureeyeNatureEyeExplore} employ \gls{ntn}-enabled \glspl{uav} to provide live aerial tours of scenic locations, broadening the scope of \gls{vr} applications and enhancing user engagement.

As \gls{ntn} technology continues to evolve, the integration of \gls{ai} within \glspl{uav} presents opportunities for enhanced operational efficiency and autonomy. \gls{ai} algorithms can optimize flight paths, improve communication protocols, and enhance decision-making processes for \glspl{uav} in real-time. Companies like Skydio \autocite{skydioSkydioAutonomous} leverage \gls{ai} to enable their \glspl{uav} to autonomously navigate complex environments and avoid obstacles. Moreover, \gls{ml} techniques can enhance \glspl{uav}' ability to analyze data collected from their surroundings. Notably, researchers from the University of Zurich developed an autonomous drone system called Swift \autocite{Kaufmann2023ChampionlevelDR}, which can outperform human champions in \gls{fpv} drone racing. This achievement represents a significant milestone, highlighting the potential of \gls{ai} in real-time decision-making and its applications in sectors such as environmental monitoring and precision agriculture.

In conclusion, the modern trends of \glspl{ntn} in \gls{uav} applications demonstrate the transformative impact these technologies can have across various sectors. From enhancing emergency response capabilities to revolutionizing communication infrastructure, \gls{ntn}-enabled \glspl{uav} are poised to play a critical role in shaping the future of wireless communication. Continued research and innovation in this field will unlock new possibilities and applications, ultimately contributing to the advancement of non-terrestrial networks and their integration with terrestrial systems.
