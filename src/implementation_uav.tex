\section{Unmanned Aerial Vehicle}\label{sec:implementation_uav}

The \gls{uav} is the base platform and the main component of the system. For this reason, it is the first component to be implemented. The \gls{uav} is responsible for carrying the reconnaissance platform and the communication system, as well as for executing the flight plan generated by the ground station. The \gls{uav} has multiple subsystems that must be implemented and integrated, in the next subsections, the implementation of each subsystem is detailed as well as the integration of the subsystems into the \gls{uav}.

\subsection{Airframe}\label{subsec:implementation_airframe}

For the airframe, the \gls{uav} was built using following the instructions provided by the manufacturer. However, some modifications were made to the airframe to accommodate the additional components using custom 3D printed parts (e.g., the landing gear, the camera mount, and the payload bay). The design and manufacture of the 3D printed parts where made using the \todo{add software} software and a \todo{add printer} 3D printer. The reason to use 3D printed parts is that they are easy to design and manufacture, as well as being lightweight and durable. The 3D printed parts were designed to be easily attached to the airframe using screws and nuts, as well as to be easily removed in case of maintenance or replacement. Some of the 3D printed parts used in the airframe can be seen in \todo{add figure}.

\todo{add figure of the 3D printed parts}

The final assemply of the airframe can be seen in \todo{add figure}. The final weight of the \gls{uav} was \todo{add weight} and the dimensions were \todo{add dimensions}.

\todo{add figure of the final assembly of the airframe}

\subsection{Propulsion System}\label{subsec:implementation_propulsion_system}

Regarding the propulsion system, the \gls{uav} was equipped with four \todo{add ref to where we bought them} motors, four \todo{add ref to where we bought them} propellers, and four \todo{add ref to where we bought them} electronic speed controllers, as stated in the design chapter, refer to \cref{subsec:design_propulsion_system}. The motors were attached to the airframe using custom metal brackets, as seen in \todo{add figure}. The electronic speed controllers were attached to the airframe using \todo{add how they were attached}, refer to \todo{add figure}.

\todo{add figure of the motors attached to the airframe}

\todo{add figure of the electronic speed controllers attached to the airframe}

\subsection{Flight Controller}\label{subsec:implementation_flight_controller}

The \gls{uav} was equipped with a \todo{add ref to where we bought them} flight controller. The flight controller was attached to the airframe using \todo{add how it was attached}, refer to \todo{add figure}. The flight controller was connected the different subsystems of the \gls{uav} using \todo{add how they were connected}. The main software used to configure the flight controller is the \todo{add software} software, as it provides a user-friendly interface to configure the different parameters of the flight controller. Moreover, it is the defacto standard for \gls{uav} flight controllers for hobbyists and professionals, with a large community of developers \todo{ref} and a large repository of documentation and tutorials \todo{add ref to the documentation}.

\todo{add figure of the flight controller attached to the airframe}

\subsection{Power System}\label{subsec:implementation_power_system}

Powering the \gls{uav} is a critical aspect of the design, as the \gls{uav} must be able to fly for a long period of time to perform the reconnaissance tasks. Moreover, the power system must be reliable and safe, as any failure in the power system can result in the loss of the \gls{uav} and it must be able to power up the different subsystems of the \gls{uav}. Regarding the power system, the \gls{uav} was equipped with a \todo{add ref to where we bought them} battery, a \todo{add ref to where we bought them} power distribution board, and a \todo{add ref to where we bought them} voltage regulator, as stated in the design chapter, refer to \cref{subsec:design_power_system}. The battery was attached to the airframe using \todo{add how it was attached}, refer to \todo{add figure}. The power distribution board was attached to the airframe using \todo{add how it was attached}, refer to \todo{add figure}. The voltage regulator was attached to the airframe using \todo{add how it was attached}, refer to \todo{add figure}.

\todo{add figure of the battery attached to the airframe}

\todo{add figure of the power distribution board attached to the airframe}

\todo{add figure of the voltage regulator attached to the airframe}

\subsection{Peripherals}\label{subsec:implementation_peripherals}

The \gls{uav} was equipped with the following peripherals, a \todo{add gps type} \gls{gps} module mounted on the top of the airframe with a 3D printed mount to provide with the best reception, and a kill switch mounted on the side of the airframe. The \gls{gps} module was connected to the flight controller using \todo{add how they were connected}, refer to \todo{add figure}. The kill switch was connected to the flight controller using \todo{add how they were connected}, refer to \todo{add figure}.

\todo{add figure of the GPS module attached to the airframe}

\todo{add figure of the kill switch attached to the airframe}

Furthermore, the \gls{uav} was equipped with a \todo{add ref to where we bought them} camera mounted on the front of the airframe with a 3D printed mount to provide with the best view. The camera was connected to the reconnaissance platform using \todo{add how they were connected}, refer to \todo{add figure}. The camera was used to capture images and videos of the environment, as well as to provide the reconnaissance platform with the data needed to detect and track objects in the environment.
