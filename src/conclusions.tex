\chapter{Conclusions}\label{ch:conclusions}

This thesis presents a comprehensive approach to developing and implementing an autonomous drone system for \gls{ntn} in remote areas. The project successfully achieved its objectives of creating a modular, cost-effective, and open-source solution that can be easily adapted for various reconnaissance tasks. The implemented system demonstrates the potential of integrating \glspl{uav} with advanced communication technologies and artificial intelligence to enhance connectivity and data collection in challenging environments.

The designed \gls{uav} platform, equipped with a 360-degree camera and onboard processing capabilities, proved capable of autonomous flight and real-time object detection. The integration of a secure communication system, combining \gls{rf} and \gls{4g} technologies, enabled reliable data transmission between the drone and the control station. Furthermore, the development of a sophisticated reconnaissance platform, leveraging deep learning algorithms and \glspl{llm}, showcased the system's ability to provide valuable insights from collected data.

The project's modular approach and use of commercially available components make it accessible for other researchers and organizations to replicate and build upon. This aligns well with the goal of fostering new research in the field of non-terrestrial networks and low-altitude drone-based systems. The successful implementation and testing of the system in a controlled environment demonstrated its potential for applications in humanitarian aid, environmental monitoring, and disaster response.

While some challenges were encountered during the development process, particularly in component selection and system integration, these were ultimately overcome through iterative design and testing. The final system met all specified requirements, including flight duration, autonomous operation, and real-time data processing capabilities.

In conclusion, this thesis contributes a valuable proof-of-concept for autonomous drone systems in non-terrestrial networks, paving the way for future advancements in this field. The developed platform offers a flexible foundation for further research and practical applications in remote area connectivity and reconnaissance.
