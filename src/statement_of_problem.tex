% As of 2024-10-20, this part has been completed and reviewed by Andres

\chapter{Statement of the problem}\label{ch:statement_of_problem}

Current research in the domain of \gls{ntn} primarily focuses on the development of satellite constellations in \gls{geo} and \gls{leo} \autocite{non_terrestial_networks_trends}. Despite the fact that these satellites offer significant potential, their high costs and limited customizability render them impractical for many research groups and individual researchers. Moreover, the complexity of launching and maintaining satellites in orbit presents a significant barrier to entry for many interested parties.

This work seeks to address the gap by developing a comprehensive, open-source, and cost-effective solution for the deployment of drones in remote environments to study and research \glspl{ntn}. Moreover, this solutions seeks to open the door to a new area of research in the domain of \gls{ntn}, low altitude drone-based networks, and to provide a platform for researchers and individuals to explore other potential applications of drones in remote areas.

Given the broad nature of this problem, the scope of this thesis will be narrowed to a specific environment and use case, as outlined below:

\begin{itemize}
  \item \textbf{Environment:} The modeled environment will be a remote area with minimal infrastructure, such as a forest, desert, or mountain. Specifically, the case study will involve an esplanade—a flat area devoid of significant obstacles like buildings or trees—allowing the drone to operate without the risk of collision. Moreover, \gls{4g} or \gls{5g} connectivity will be assumed to be available, enabling the drone to communicate with a ground station.

  \item \textbf{Atmospheric Conditions:} The selected environment will feature a clear sky, with minimal electromagnetic interference from other sources such as drones or aircraft. Additionally, the conditions will approximate \gls{stp} of \SI{15}{\degreeCelsius} and \SI{1013}{\hecto\pascal}.

  \item \textbf{Operational Parameters:} The drone operations will be confined to \gls{vlos} and an altitude below \SI{120}{\metre} and the \gls{mtow} of the drone will not exceed \SI{25}{\kilogram}, ensuring compliance with current aviation regulations in Spain and most countries.

  \item \textbf{Hardware:} The drone will be constructed using commercially available, off-the-shelf components to ensure affordability and ease of replication for other research groups and individuals.
\end{itemize}

% Local Variables:
% jinx-local-words: "customizability mtow ntn terrestial"
% End:
