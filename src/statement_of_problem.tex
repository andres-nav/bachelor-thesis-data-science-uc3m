\chapter{Statement of the problem}
\label{ch:statement_of_problem}

Current research in the domainof \gls{ntn} primarily focuses on the development of satellite constellations in \gls{geo} and \gls{leo} \autocite{non_terrestial_networks_trends}. While these satellites offer significant potential, their high costs and limited customizability render them impractical for many research groups and individual researchers. Moreover, the complexity of launching and maintaining satellites in orbit presents a significant barrier to entry for many interested parties. In contrast, drones present a more affordable and adaptable solution for rapid deployment of communication networks in remote areas as well as low-altitude surveillance and monitoring missions.

The core challenge addressed in this thesis is the lack of widely accessible solutions for the deployment of autonomous drones in remote areas. Presently, available solutions are either expensive, proprietary systems that lack customization, or they require extensive infrastructure, such as ground stations or high-speed internet connectivity. These requirements severely limit the application of drones in areas with minimal or no infrastructure.

This work seeks to address the gap by developing a comprehensive, open-source, and cost-effective solution for autonomous drone missions in remote environments. Given the broad nature of this problem, the scope of this thesis will be narrowed to a specific environment and use case, as outlined below:

\begin{itemize}
  \item \textbf{Environment:} The modeled environment will be a remote area with minimal infrastructure, such as a forest, desert, or mountain. Specifically, the case study will involve an esplanade—a flat area devoid of significant obstacles like buildings or trees—allowing the drone to operate without the risk of collision. Moreover, \gls{4g} or \gls{5g} connectivity will be assumed to be available, enabling the drone to communicate with a ground station.

  \item \textbf{Atmospheric Conditions:} The selected environment will feature a clear sky, with minimal electromagnetic interference from other sources such as drones or aircraft. Additionally, the conditions will approximate \gls{stp} of \SI{15}{\degreeCelsius} and \SI{1013}{\hecto\pascal}.

  \item \textbf{Operational Parameters:} The drone operations will be confined to \gls{vlos} and an altitude below \SI{120}{\metre}, ensuring compliance with current aviation regulations in Spain and most countries. The \gls{mtow} of the drone will not exceed \SI{25}{\kilogram}.

  \item \textbf{Hardware:} The drone will be constructed using commercially available, off-the-shelf components to ensure affordability and ease of replication for other research groups and individuals.

  \item \textbf{Application:} This work will focus on the application of drones in human rescue missions, such as search and rescue operations in disaster-stricken areas or monitoring remote environments for signs of danger. While there are other potential applications of drones in remote areas, human rescue offers significant societal benefits.
\end{itemize}
