\section{Reconnaissance Platform}\label{sec:design_reconnaissance_platform}

The reconnaissance platform is the system that processes the data collected by the peripherals and the sensors on the \gls{uav} and provides insights to the end-user for the different reconnaissance tasks. The reconnaissance platform is responsible detecting and tracking objects, monitoring the environment, and generating alerts and notifications in case of critical events or failures. The reconnaissance platform is also responsible for managing the different missions of the \glspl{uav} with the objectives and constraints defined by the end-user.

The reconnaissance platform is divided into two main components. The on-board reconnaissance platform, which runs on the \gls{uav}, is responsible for processing the data collected by the peripherals and the sensors on the \gls{uav}, detecting the objects in the environment, and providing insights to the off-board reconnaissance platform. The off-board reconnaissance platform, which runs on a server, is responsible for coordinating the missions of the \glspl{uav}, collecting the data of the \glspl{uav}, and generating reports and alerts for the end-user.

For the on-board reconnaissance platform, the components chosen were the on-board computer, specifically the NVIDIA Jetson Orin \autocite{nvidiaNVIDIAJetson} as it provides the best real-time processing power commercially available to run deep learning algorithms, \SI{275}{TOPS}. The main reason to choose this on-board computer enables the \gls{uav} to process the data collected in real-time and reduces the bandwidth required to send the data to the off-board reconnaissance platform. Moreover, a 360-degree camera was chosen to provide a full view of the environment where the \gls{uav} is operating and be able to detect and track objects in real-time regardless of the \gls{uav}'s orientation. The camera chosen was theRicoh Theta X 360 Degree Camera \autocite{ricohimagingTHETARicoh}. The reasoning to choose this camera is the only few 360-degree cameras that provide live streaming capabilities and a high resolution, \SI{5.7}{K} at \SI{30}{FPS}.

The off-board reconnaissance platform is designed to run on any available server, as it is not required to have a high processing power as the on-board reconnaissance platform. The main requirement for the off-board reconnaissance platform is to have a stable and reliable connection to the on-board reconnaissance platform, as well as the ability to be accessed remotely from anywhere in the world. For this, a cloud provider was chosen, specifically Amazon Web Services \autocite{amazonCloudComputing}.

% Local Variables:
% jinx-local-words: "uav"
% End:
