\chapter{Methodology Approach}

\todo{need to rephrasea all this}
The methodology approach used in this thesis is based on the V-model as outlined by the International Council of System Engineering (INCOSE) standard for project development. The V-model is a rigorous and structured approach to project development that ensures that all aspects of the project are considered and that the project is completed on time and within budget. It is achieived by a thorough development process, facilitating clear validation and verification of initial requirements at each stage.

The methodology is divided into seven parts:

\begin{enumerate}
  \item Identification of Solution Requirements: Before taking any actions, it is crucial to understand what can be achieved and the rationale behind it. To this end, the author has focused on deriving valuable insights from the reviewed literature in the State of the Art (SOT A) and identifying critical scenarios. Subsequently, and in conjunction with the thesis objectives and the applicable regulatory framework, a list of solution requirements is created. This list will later serve as a benchmark for validating the proposed solution.

  \item Identification of System Requirements: Technical requirements are formulated to meet the previously established solution requirements. This includes a high-level overview of the proposed solution’s components, the rationale for their selection, and their interconnections.

  \item Identification of Component Requirements: Building on the high-level architecture of the solution, a more detailed approach is outlined for each component, considering their specific power and data transmission needs. This results in a detailed architecture of the solution.

  \item Manufacturing and Implementation: The proposed solution is manufactured using available tools, while simultaneously integrating the necessary electrical components.

  \item Component Verification: Each component’s functionality is verified in standalone mode, with detailed information provided on the verification process.

  \item Integration Testing and Flight Testing: The methodology for conducting flight tests and post-flight analysis is described. System integration is performed by checking communication between module pairs to ensure that data can travel freely and be used effectively.

  \item Deployment and Preparation for Future Upgrades: Validation of the initial requirements is conducted to ensure that all solution requirements have been addressed. This step also includes preparations for potential future upgrades.
\end{enumerate}
