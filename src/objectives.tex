\chapter{Objectives}\label{ch:objectives}

The primary aim of this thesis is to provide the research community and humanitarian organizations with an open-source, modular, and customizable drone capable of operating in remote areas. The drone will be designed to be cost-effective, user-friendly, and competitive with other market offerings. It will incorporate the latest advancements in drone technology to ensure optimal performance and reliability.

In addition, a software platform will be developed to enable the operation of the drone for reconnaissance tasks. The platform will support multiple drones, allowing for the deployment of a coordinated swarm to conduct surveillance and monitoring missions effectively. The software will be user-friendly, enabling researchers and humanitarian organizations to program the drones for specific tasks easily.

To achieve these objectives, the following specific aims will be pursued:

\begin{itemize}
  \item The design must be modular and customizable, enabling easy modifications to adapt the drone for different applications.

  \item The components utilized in the drone should be off-the-shelf and readily available, facilitating straightforward assembly and repairs.

  \item The drone must be capable of autonomous flight to enable operations in remote areas where manual control is challenging.

  \item The design will incorporate the latest advancements in drone technology, ensuring competitiveness with other market offerings.

  \item The drone must be capable of communicating with a ground station to facilitate remote control.

  \item The software platform should be programmable to perform specific tasks, such as reconnaissance of designated areas and monitoring of particular parameters.

  \item The software platform must support multiple drones, allowing for the deployment of a coordinated swarm to conduct reconnaissance tasks effectively.
\end{itemize}
