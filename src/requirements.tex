\chapter{Requirements}

Based on careful analysis of the conclusions from the current trends in \glspl{uav} outlined in \cref{ch:modern_trends} and the objectives reviewed in \cref{ch:objectives}, the following requirements are established for the high-level system as well as the detailed requirements for the \gls{uav}, control station, and software platform.

\section{High-level System Requirements}

The high-level system requirements are as follows:

\begin{itemize}
  \item The system must be able to operate in remote areas with limited infrastructure, such as roads, electricity, and internet connectivity.

  \item The system must be able to be monitored remotely, with the ability to communicate with a ground station via a \gls{4g} or \gls{3g} connection.

  \item The system must be cost-effective, with the ability to be assembled and disassembled easily, and to be repaired and maintained with minimal effort.

  \item The system must be modular, allowing for the integration of different sensors and payloads for different applications, as well as, the scalability of the system to include multiple \glspl{uav} working together in a coordinated manner.

  \item The system must be able to perform reconnaissance tasks autonomously, with the ability to take off, land, and navigate given a set of waypoints.

  \item The system must comply with the applicable regulatory framework for \glspl{uav} in the country of operation, Spain, as well as the \gls{eu} regulations. See \cref{ch:regulatory_framework} for more information.
\end{itemize}

\section{Unmanned Aerial Vehicle Requirements}

The \gls{uav} requirements are as follows:

\begin{itemize}
  \item The \gls{uav} must be able to be controlled remotely, with the ability to communicate with a ground station in real-time.

  \item The \gls{uav} must be able to take off, land, and navigate autonomously, with the ability to update its flight plan in real-time.

  \item The \gls{uav} must be able to process data in real-time, with the ability to relay the information to the ground station.

  \item The \gls{uav} must be able to carry different payloads and sensors for different applications up to a maximum payload weight of 2 kg, with the ability to adapt to different reconnaissance tasks.

  \item The \gls{uav} must be able to fly for a minimum of 30 minutes, without the need for recharging.

  \item The \gls{uav} must be have a failsafe mechanism, that is it must be able to return to the ground station in case of loss of communication or other critical failures.

  \item The \gls{uav} must be able to keep a fixed altitude and position.

  \item The \gls{uav} must comply with the EASA regulations for the Open Category, with a maximum limit set at 25 kg of MTOW and 3 meters of wingspan.

  \item The \gls{uav} must be able to perform reconnaissance tasks, such as mapping, surveillance, and monitoring the environment.
\end{itemize}

\section{Control Station Requirements}

The control station requirements are as follows:


\begin{itemize}
  \item The control station must be able to receive telemetry data from the \gls{uav} in real-time, with the ability to send commands to the \gls{uav} to update its flight plan.

  \item The control station must be able to be used remotely, with the ability to communicate with the \gls{uav} via a \gls{4g} or \gls{3g} connection.

  \item The control station must be able to create a geofence around the area of operation, with the ability to monitor the \gls{uav}'s position and altitude in real-time.

  \item The control station must have the capability be able to track multiple \glspl{uav} simultaneously, with the ability to coordinate their flight plans and tasks.

  \item The control station must log all telemetry data and flight information, with the ability to analyze the data and generate reports.
\end{itemize}

\section{Software Platform Requirements}

The software platform requirements are as follows:

\begin{itemize}
  \item The software platform must be able to run on a variety of operating systems, with the ability to communicate with the \glspl{uav} and the control station in real-time.

  \item The software platform must be able to be used remotely, with the ability to access the \glspl{uav} and the control station via a \gls{4g} or \gls{3g} connection.

  \item The software platform must be reliable, secure, and easy to use, allowing for the programming of the \glspl{uav} to perform specific tasks and the coordination of multiple \glspl{uav} in a swarm.

  \item The software platform must be customizable, allowing for the integration of new features and the modification of existing ones, as well as, the addition of new \glspl{uav} to the system and different types of reconnaissance tasks.

  \item The software platform must have alerting and notification capabilities, with the ability to send alerts and notifications to the user in case of critical events or failures.

  \item The software platform must have a user-friendly interface, with the ability to display telemetry data and flight information in real-time, as well as, the ability to monitor the \glspl{uav} in real-time.
\end{itemize}
