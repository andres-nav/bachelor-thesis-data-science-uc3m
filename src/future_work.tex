\chapter{Future work}\label{ch:future_work}

In the future work for this project, several key areas of development have been identified to enhance the system's capabilities and performance. Real-time location tracking will be improved by developing a robust communication protocol between the on-board computer and the off-board server, enabling precise monitoring of the UAV's position throughout its mission. This advancement will significantly enhance the system's ability to track and manage drone operations in real-time. The implementation of swarm intelligence algorithms represents another exciting avenue for future research, allowing multiple UAVs to work collaboratively and efficiently cover larger areas while performing complex tasks. This development could dramatically increase the system's overall effectiveness and scalability.

Further integration of advanced AI capabilities, such as predictive analytics and anomaly detection, will be pursued to enhance the system's autonomous identification of patterns and potential threats. This improvement will make the system more proactive and intelligent in its decision-making processes. To expand the system's operational range, efforts will be made to develop weatherproofing solutions, enabling the UAV to function effectively in various environmental conditions. This enhancement will significantly increase the system's versatility and usability across different climates and terrains. The integration of additional sensors, including thermal cameras and LIDAR, will be explored to broaden the system's detection and analysis capabilities across different spectrums and conditions, providing a more comprehensive understanding of the surveyed environment. Finally, optimizing the on-board processing capabilities for edge computing will be a priority, allowing the system to handle more complex computations locally. This optimization will reduce the need for constant communication with the off-board server, improving real-time performance and efficiency.
